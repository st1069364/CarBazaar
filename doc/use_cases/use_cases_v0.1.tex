%% Overleaf			
%% Software Manual and Technical Document Template	
%% 									
%% This provides an example of a software manual created in Overleaf.

\documentclass{../ol-softwaremanual}

% Packages used in this example
\usepackage{graphicx}  % for including images
\usepackage{microtype} % for typographical enhancements
\usepackage{minted}    % for code listings
\usepackage{amsmath}   % for equations and mathematics
\setminted{style=friendly,fontsize=\small}
\renewcommand{\listoflistingscaption}{List of Code Listings}
\usepackage{hyperref}  % for hyperlinks
\usepackage[a4paper,top=4.2cm,bottom=4.2cm,left=3.5cm,right=3.5cm]{geometry} % for setting page size and margins

\usepackage[english, greek]{babel}

\usepackage{subfig}

\usepackage{incgraph,tikz}

\usepackage{filemod}





\usepackage{rotating}


% Custom macros used in this example document
\newcommand{\doclink}[2]{\href{#1}{#2}\footnote{\url{#1}}}
\newcommand{\cs}[1]{\texttt{\textbackslash #1}}

\begin{document}
	
	
	\begin{titlepage}
		
		
		% Frontmatter data; appears on title page
		\title{\en Use Cases \\}
		\version{0.1}
		\softwarelogo{\includegraphics[scale=0.4]{../CarBazaar_logo.png}}		
		
	\end{titlepage}
	
	
	\maketitle
	
	\newpage
	
	\center{\textbf{Μέλη Ομάδας}}
	
	\vspace{20pt}
	
	
	
	\begin{table}[htbp!]
		
		\begin{tabular}{llll}
			Μεμελετζόγλου Χαρίλαος & 1069364 & \en st1069364@ceid.upatras.gr & 4o Έτος   \\ 
			\\ Λέκκας Γεώργιος      &      1067430    &   \en st1067430@ceid.upatras.gr & 4o Έτος  \\
			\\ Γιαννουλάκης Ανδρέας        &   1067387       & \en st1067387@ceid.upatras.gr & 4o Έτος           \\
			\\ Κανελλόπουλος Ιωακείμ        &  1070914        &    \en st1070914@ceid.upatras.gr & 4o Έτος        \\ 
		\end{tabular}
	\end{table}
	
	\center{\textbf{Υπεύθυνοι Παρόντος Τεχνικού Κειμένου}}
	
	\vspace{20pt}
	
	\begin{table}[htbp!]
		\begin{tabular}{ll}
			Μεμελετζόγλου Χαρίλαος & \en Editor \\
			\\ Λέκκας Γεώργιος      &   \en  Contributor \\
			\\ Γιαννουλάκης Ανδρέας & \en Contributor \\
			\\ Κανελλόπουλος Ιωακείμ & \en Contributor \\ 
		\end{tabular}
	\end{table}
	
	
	\vspace{20pt}
	
	\center{\textbf{Εργαλεία που χρησιμοποιήθηκαν}}
	
	\vspace{20pt}
	\flushleft
	Χρησιμοποιήθηκε το \en \doclink{https://www.overleaf.com/}{Overleaf} \gr και το \en \doclink{https://www.texstudio.org/}{TexStudio} \gr για την συγγραφή του \LaTeX\ κώδικα. \break
	
	Για την δημιουργία του λογότυπου, χρησιμοποιήθηκε το εργαλείο \en \doclink{https://www.adobe.com/express/create/logo}{Adobe Express} . \gr \break
	
	\newpage
	
	\center{\textbf{\en Use Cases \gr}}
	
	\paragraph{\en Use Case 1: \gr Ανάρτηση Αγγελίας Πώλησης Μεταχειρισμένου Οχήματος}
	
	\begin{enumerate}
		
		\item Ο χρήστης επιλέγει \en " \gr Ανάρτηση Αγγελίας \en " \gr 
		\item Το σύστημα αποστέλλει στο \en email \gr του χρήστη, έναν κωδικό ασφαλείας
		\item Ο χρήστης εισάγει τον, απεσταλμένο από το σύστημα, κωδικό
		\item Ο χρήστης εισάγει τα στοιχεία επικοινωνίας του και την τοποθεσία του
		\item Ο χρήστης εισάγει τον τίτλο της αγγελίας
		\item Ο χρήστης εισάγει στοιχεία του οχήματος όπως μάρκα, μοντέλο, έτος κυκλοφορίας, χιλιόμετρα
		\item Το σύστημα ελέγχει πως όντως κυκλοφορεί αντίστοιχο μοντέλο αυτοκινήτου στην αγορά
		\item Το σύστημα εμφανίζει στον χρήστη μια εκτίμηση της τιμής του οχήματος, με βάση την κατάστασή του και ρωτά τον χρήστη αν αποδέχεται την συγκεκριμένη τιμή
		\item Ο χρήστης συναινεί στην προτεινόμενη τιμή
		\item Ο χρήστης προσθέτει περιγραφή και περαιτέρω πληροφορίες για το όχημα
		\item Ο χρήστης αναρτά φωτογραφίες του οχήματος
		\item Το σύστημα δημιουργεί το \en 3D \gr μοντέλο του οχήματος
		\item Το σύστημα δημιουργεί την αγγελία και εμφανίζει μια προεπισκόπηση στον χρήστη
		\item Ο χρήστης εγκρίνει την αγγελία και προχωρά στην δημοσίευσή της		
	\end{enumerate}
	
	\paragraph{Εναλλακτική Ροή 1}
	
	\begin{enumerate}
		\item O χρήστης εισάγει στοιχεία μη-υπαρκτού μοντέλου
		\item Το σύστημα ενημερώνει τον χρήστη σχετικά με τα λανθασμένα πεδία που έχει συμπληρώσει
		\item Ο χρήστης προβαίνει στις απαραίτητες διορθώσεις και η Περίπτωση Χρήσης συνεχίζει από το βήμα 8 της βασικής ροής
	\end{enumerate}

	\paragraph{Εναλλακτική Ροή 2}
	
	\begin{enumerate}
		\item O χρήστης εισάγει διαφορετικό κωδικό ασφαλείας από αυτόν που στάλθηκε στο \en email \gr  του
		\item Το σύστημα ενημερώνει τον χρήστη για το λάθος
		\item Ο χρήστης εισάγει τον σωστό κωδικό ασφαλείας και η Περίπτωση Χρήσης προχωρά από το βήμα 4 της βασικής ροής
	\end{enumerate}

	\paragraph{Εναλλακτική Ροή 3}
	
	\begin{enumerate}
		\item O χρήστης δεν συναινεί με την προτεινόμενη από το σύστημα τιμή πώλησης του οχήματος
		\item O χρήστης εισάγει μια διαφορετική τιμή και η Περίπτωση Χρήσης συνεχίζει από το βήμα 10 της βασικής ροής		 
	\end{enumerate}

	\paragraph{Εναλλακτική Ροή 4}
	
	\begin{enumerate}
		\item Ο χρήστης δεν εισάγει περιγραφή και λεπτομερής πληροφορίες για το όχημα
		\item Το σύστημα προειδοποιεί τον χρήστη και τον προτρέπει να συμπληρώσει τα αντίστοιχα πεδία κειμένου
		\item Ο χρήστης εισάγει τις απαραίτητες ελλείπουσες πληροφορίες και η Περίπτωση Χρήσης συνεχίζει από το βήμα 11 της βασικής ροής
	\end{enumerate}
	
	
	\paragraph{\en Use Case 2: \gr Προγραμματισμός Ελέγχου Οχήματος}
	
	\begin{enumerate}
		\item Ο χρήστης επιλέγει \en " \gr Έλεγχος Οχήματος \en " \gr
		\item Ο χρήστης επιλέγει το πακέτο ελέγχου που επιθυμεί και την ημερομηνία και ώρα
		\item Ο χρήστης εισάγει την περιοχή του
		\item Το σύστημα ρωτά τον χρήστη αν επιθυμεί να επιλέξει ελεγκτή της αρεσκείας του ή να του προταθεί κάποιος αυτόματα με βάση την τοποθεσία του
		\item Ο χρήστης επιλέγει να ορίσει αυτός τον ελεγκτή και εισάγει τα στοιχεία του
		\item Ο χρήστης εισάγει τα στοιχεία του οχήματος και επιλέγει αν επιθυμεί την έκδοση πιστοποιητικών εγγράφων σχετικά με την κατάσταση του οχήματος
		\item Το σύστημα εμφανίζει την τελική τιμή του ελέγχου καθώς και την διάρκειά του
		\item Ο χρήστης επιλέγει τον τρόπο πληρωμής και προχωρά στην πληρωμή 
		\item Το σύστημα εμφανίζει μήνυμα επιτυχούς κράτησης και αποστέλλει \en email \gr στον χρήστη, με τα στοιχεία του ραντεβού και του ελεγκτή		
	\end{enumerate}

	\paragraph{Εναλλακτική Ροή 1}
	
	\begin{enumerate}
		\item Ο χρήστης εισάγει μη-υπαρκτό Ταχυδρομικό Κώδικα
		\item Το σύστημα ενημερώνει τον χρήστη με το κατάλληλο μήνυμα σφάλματος και τον προτρέπει να χρησιμοποιήσει την υπηρεσία Εντοπισμού Τοποθεσίας
		\item Ο χρήστης ενεργοποιεί την υπηρεσία 
		\item Το σύστημα εντοπίζει τον χρήστη και η Περίπτωση Χρήσης προχωρά από το βήμα 4 της βασικής ροής
	\end{enumerate}

	\paragraph{Εναλλακτική Ροή 2}
	
	\begin{enumerate}
		\item Ο χρήστης δεν εισάγει ελεγκτή και επιτρέπει στο σύστημα να του προτείνει ελεγκτές με βάση την τοποθεσία του
		\item Το σύστημα εμφανίζει μια λίστα με τους ελεγκτές που βρίσκονται στην περιοχή του χρήστη
		\item Ο χρήστης επιλέγει ελεγκτή και η Περίπτωση Χρήσης συνεχίζει από το βήμα 6 της βασικής ροής
	\end{enumerate}
	
	
\end{document}
 
