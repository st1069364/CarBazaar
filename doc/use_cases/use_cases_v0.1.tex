%% Overleaf			
%% Software Manual and Technical Document Template	
%% 									
%% This provides an example of a software manual created in Overleaf.

\documentclass{../ol-softwaremanual}

% Packages used in this example
\usepackage{graphicx}  % for including images
\usepackage{microtype} % for typographical enhancements
\usepackage{minted}    % for code listings
\usepackage{amsmath}   % for equations and mathematics
\setminted{style=friendly,fontsize=\small}
\renewcommand{\listoflistingscaption}{List of Code Listings}
\usepackage{hyperref}  % for hyperlinks
\usepackage[a4paper,top=4.2cm,bottom=4.2cm,left=3.5cm,right=3.5cm]{geometry} % for setting page size and margins

\usepackage[english, greek]{babel}

\usepackage{subfig}

\usepackage{incgraph,tikz}

\usepackage{filemod}





\usepackage{rotating}


% Custom macros used in this example document
\newcommand{\doclink}[2]{\href{#1}{#2}\footnote{\url{#1}}}
\newcommand{\cs}[1]{\texttt{\textbackslash #1}}

\begin{document}
	
	
	\begin{titlepage}
		
		
		% Frontmatter data; appears on title page
		\title{\en Use Cases \\}
		\version{0.1}
		\softwarelogo{\includegraphics[scale=0.4]{../CarBazaar_logo.png}}		
		
	\end{titlepage}
	
	
	\maketitle
	
	\newpage
	
	\center{\textbf{Μέλη Ομάδας}}
	
	\vspace{20pt}
	
	
	
	\begin{table}[htbp!]
		
		\begin{tabular}{llll}
			Μεμελετζόγλου Χαρίλαος & 1069364 & \en st1069364@ceid.upatras.gr & 4o Έτος   \\ 
			\\ Λέκκας Γεώργιος      &      1067430    &   \en st1067430@ceid.upatras.gr & 4o Έτος  \\
			\\ Γιαννουλάκης Ανδρέας        &   1067387       & \en st1067387@ceid.upatras.gr & 4o Έτος           \\
			\\ Κανελλόπουλος Ιωακείμ        &  1070914        &    \en st1070914@ceid.upatras.gr & 4o Έτος        \\ 
		\end{tabular}
	\end{table}
	
	\center{\textbf{Υπεύθυνοι Παρόντος Τεχνικού Κειμένου}}
	
	\vspace{20pt}
	
	\begin{table}[htbp!]
		\begin{tabular}{ll}
			Μεμελετζόγλου Χαρίλαος & \en Editor \\
			\\ Λέκκας Γεώργιος      &   \en  Editor \\
			\\ Γιαννουλάκης Ανδρέας & \en Contributor \\
			\\ Κανελλόπουλος Ιωακείμ & \en Contributor \\ 
		\end{tabular}
	\end{table}
	
	
	\vspace{20pt}
	
	\center{\textbf{Εργαλεία που χρησιμοποιήθηκαν}}
	
	\vspace{20pt}
	\flushleft
	Χρησιμοποιήθηκε το \en \doclink{https://www.overleaf.com/}{Overleaf} \gr και το \en \doclink{https://www.texstudio.org/}{TexStudio} \gr για την συγγραφή του \LaTeX\ κώδικα. \break
	
	Για την δημιουργία του λογότυπου, χρησιμοποιήθηκε το εργαλείο \en \doclink{https://www.adobe.com/express/create/logo}{Adobe Express} . \gr \break
	
	\newpage
	
	\center{\textbf{\en Use Cases \gr}}
	
	\paragraph{\en Use Case 1: \gr Ανάρτηση Αγγελίας Πώλησης Μεταχειρισμένου Οχήματος}
	
	\begin{enumerate}
		
		\item Ο χρήστης επιλέγει \en " \gr Ανάρτηση Αγγελίας \en " \gr 	
		\item Ο χρήστης εισάγει την τοποθεσία του καθώς και τον τίτλο της αγγελίας
		\item Ο χρήστης εισάγει στοιχεία του οχήματος όπως μάρκα, μοντέλο, έτος κυκλοφορίας, χιλιόμετρα, κυβικά, τύπος καυσίμου, χρώμα, αριθμός πινακίδας, κλπ
		\item Το σύστημα ελέγχει πως όντως κυκλοφορεί αντίστοιχο μοντέλο αυτοκινήτου στην αγορά
		\item Το σύστημα ρωτά τον χρήστη αν επιθυμεί να ανεβάσει αρχεία πιστοποίησης της κατάστασης του οχήματος
		\item Ο χρήστης ανεβάζει τα απαραίτητα αρχεία που έχουν προκύψει από τον έλεγχο του οχήματος		
		\item Το σύστημα εμφανίζει στον χρήστη μια εκτίμηση της τιμής του οχήματος, με βάση την κατάστασή του και ρωτά τον χρήστη αν αποδέχεται την συγκεκριμένη τιμή
		\item Ο χρήστης συναινεί στην προτεινόμενη τιμή
		\item Ο χρήστης προσθέτει περιγραφή και περαιτέρω πληροφορίες για το όχημα
		\item Ο χρήστης αναρτά φωτογραφίες του οχήματος
		\item Το σύστημα δημιουργεί το \en 3D \gr μοντέλο του οχήματος
		\item Το σύστημα δημιουργεί την αγγελία και εμφανίζει μια προεπισκόπηση στον χρήστη
		\item Ο χρήστης εγκρίνει την αγγελία και προχωρά στην δημοσίευσή της		
	\end{enumerate}
	
	\paragraph{Εναλλακτική Ροή 1}
	
	\begin{enumerate}
		\item O χρήστης εισάγει στοιχεία μη-υπαρκτού μοντέλου
		\item Το σύστημα ενημερώνει τον χρήστη σχετικά με τα λανθασμένα πεδία που έχει συμπληρώσει
		\item Ο χρήστης προβαίνει στις απαραίτητες διορθώσεις και η Περίπτωση Χρήσης συνεχίζει από το βήμα 8 της βασικής ροής
	\end{enumerate}

	\paragraph{Εναλλακτική Ροή 2}
	
	\begin{enumerate}
		\item O χρήστης εισάγει διαφορετικό κωδικό ασφαλείας από αυτόν που στάλθηκε στο \en email \gr  του
		\item Το σύστημα ενημερώνει τον χρήστη για το λάθος
		\item Ο χρήστης εισάγει τον σωστό κωδικό ασφαλείας και η Περίπτωση Χρήσης προχωρά από το βήμα 4 της βασικής ροής
	\end{enumerate}

	\paragraph{Εναλλακτική Ροή 3}
	
	\begin{enumerate}
		\item O χρήστης δεν συναινεί με την προτεινόμενη από το σύστημα τιμή πώλησης του οχήματος
		\item O χρήστης εισάγει μια διαφορετική τιμή και η Περίπτωση Χρήσης συνεχίζει από το βήμα 10 της βασικής ροής		 
	\end{enumerate}

	\paragraph{Εναλλακτική Ροή 4}
	
	\begin{enumerate}
		\item Ο χρήστης δεν εισάγει περιγραφή και λεπτομερής πληροφορίες για το όχημα
		\item Το σύστημα προειδοποιεί τον χρήστη και τον προτρέπει να συμπληρώσει τα αντίστοιχα πεδία κειμένου
		\item Ο χρήστης εισάγει τις απαραίτητες ελλείπουσες πληροφορίες και η Περίπτωση Χρήσης συνεχίζει από το βήμα 11 της βασικής ροής
	\end{enumerate}
	
	
	\paragraph{\en Use Case 2: \gr Προγραμματισμός Ελέγχου Οχήματος}
	
	\begin{enumerate}
		\item Ο χρήστης επιλέγει \en " \gr Έλεγχος Οχήματος \en " \gr
		\item Ο χρήστης επιλέγει το πακέτο ελέγχου που επιθυμεί και την ημερομηνία και ώρα
		\item Ο χρήστης εισάγει την περιοχή του
		\item Το σύστημα ρωτά τον χρήστη αν επιθυμεί να επιλέξει ελεγκτή της αρεσκείας του ή να του προταθεί κάποιος αυτόματα με βάση την τοποθεσία του
		\item Ο χρήστης επιλέγει να ορίσει αυτός τον ελεγκτή και εισάγει τα στοιχεία του
		\item Ο χρήστης εισάγει τα στοιχεία του οχήματος και επιλέγει αν επιθυμεί την έκδοση πιστοποιητικών εγγράφων σχετικά με την κατάσταση του οχήματος
		\item Το σύστημα εμφανίζει την τελική τιμή του ελέγχου καθώς και την διάρκειά του
		\item Ο χρήστης επιλέγει τον τρόπο πληρωμής και προχωρά στην πληρωμή 
		\item Το σύστημα εμφανίζει μήνυμα επιτυχούς κράτησης και αποστέλλει \en email \gr στον χρήστη, με τα στοιχεία του ραντεβού και του ελεγκτή		
	\end{enumerate}

	\paragraph{Εναλλακτική Ροή 1}
	
	\begin{enumerate}
		\item Ο χρήστης εισάγει μη-υπαρκτό Ταχυδρομικό Κώδικα
		\item Το σύστημα ενημερώνει τον χρήστη με το κατάλληλο μήνυμα σφάλματος και τον προτρέπει να χρησιμοποιήσει την υπηρεσία Εντοπισμού Τοποθεσίας
		\item Ο χρήστης ενεργοποιεί την υπηρεσία 
		\item Το σύστημα εντοπίζει τον χρήστη και η Περίπτωση Χρήσης προχωρά από το βήμα 4 της βασικής ροής
	\end{enumerate}

	\paragraph{Εναλλακτική Ροή 2}
	
	\begin{enumerate}
		\item Ο χρήστης δεν εισάγει ελεγκτή και επιτρέπει στο σύστημα να του προτείνει ελεγκτές με βάση την τοποθεσία του
		\item Το σύστημα εμφανίζει μια λίστα με τους ελεγκτές που βρίσκονται στην περιοχή του χρήστη
		\item Ο χρήστης επιλέγει ελεγκτή και η Περίπτωση Χρήσης συνεχίζει από το βήμα 6 της βασικής ροής
	\end{enumerate}
	
	
    \paragraph{\en Use Case 3: \gr Αναζήτηση Ανταλλακτικού}	
    
    \begin{enumerate}
    	\item Ο χρήστης επιλέγει στο μενού της εφαρμογής την κατηγορία  \en"\gr Ανταλλακτικά \en"\gr
    	\item Το σύστημα εμφανίζει στον χρήστη μια οθόνη με πολλαπλά πεδία και φίλτρα
    	\item Ο χρήστης περιορίζει την αναζήτηση του τοποθετώντας το είδος του οχήματος,την μάρκα,το μοντέλο,τον κατασκευαστή,το εύρος τιμων,και την κατάσταση του ανταλλακτικού(καινούργιο ή μεταχειρισμένο)
    	\item Το σύστημα εμφανίζει τη λίστα με τις αγγελίες που πληρούν τα κριτήρια που έθεσε ο χρήστης
    	\item Ο χρήστης επιλέγει την αγγελία της αρεσκίας του
    	\item Το σύστημα εμφανίζει μια λεπτομερή περιγραφή του ανταλλακτικού και τα στοιχεία του πωλητή
    			
    \end{enumerate}

    \paragraph{Εναλλακτική Ροή}
	
	\begin{enumerate}
		\item Ο χρήστης δεν εισάγει χαρακτηριστικά για το ανταλλακτικό που επιθυμεί να αγοράσει
		\item Το σύστημα του εμφανίζει ανταλλακτικά για όλες τις κατηγορίες και μάρκες αυτοκινήτων και προχωρά από το βήμα 4 της βασικής ροής
	\end{enumerate}

     \paragraph{\en Use Case 4: \gr Αναζήτηση Κοντινών Αντιπροσωπειών}	
    
    \begin{enumerate}
    	\item Ο χρήστης επιλέγει στο μενού της εφαρμογής την κατηγορία \en"\gr Αντιπροσωπείες \en"\gr
    	\item Ο χρήστης τοποθετεί σε αντίστοιχο πεδίο το είδος του οχήματος που τον ενδιαφέρει
    	\item Το σύστημα εμφανίζει ένα χάρτη και ένα πεδίο αναζήτησης για αυτόν
    	\item Ο χρήστης τοποθετεί την περιοχή του  
    	\item Το σύστημα εμφανίζει όλες τις αντιπροσωπείες που βρίσκονται κοντα στην περιοχή που έθεσε ο χρήστης
    	\item Ο χρήστης επιλέγει την αντιπροσωπεία της αρεσκείας του
    	\item Το σύστημα εμφανίζει μία λίστα με οχήματα που είναι διαθέσιμα από την αντιπροσωπεία 
    	\item Ο χρήστης καταλήγει σε ένα συγκεκριμένο όχημα
    	\item Το σύστημα εμφανίζει περισσότερες πληροφορίες για το όχημα και τις λεπτομέρειες για την επικοινωνία με την αντιπροσωπεία
    	
    \end{enumerate}
    
    \paragraph{Εναλλακτική Ροή 1}
    
    \begin{enumerate}
    	\item Ο χρήστης δεν επιλέγει το είδος οχήματος που τον ενδιαφέρει
    	\item Το σύστημα πρότρέπει τον χρήστη να επιλέξει το είδος
    	\item Ο χρήστης εισάγει τα απαραίτητα στοιχεία και η περίπτωση χρήσης συνεχίζει από το βήμα 3 της βασικής ροής
    \end{enumerate}

    \paragraph{Εναλλακτική Ροή 2}
    
    \begin{enumerate}
    	\item Η περιοχή που επιλέγει στον χάρτη ο χρήστης δεν συνάδει με αυτή που είχε βάλει όταν έφτιαχνε το προφίλ του
    	\item Το σύστημα του υπενθυμίζει πως οι δύο περιοχές διαφέρουν
    	\item Ο χρήστης τοποθετεί στον χάρτη την σωστή περιοχή
    	\item  Το σύστημα εμφανίζει τις αντιπροσωπείες και η περίπτωση χρήσης συνεχίζει από το βήμα 5 της βασικής ροής
    \end{enumerate}

	\paragraph{\en Use Case 5: \gr Σύγκριση Αυτοκινήτων}
	\begin{enumerate}
		\item Ο χρήστης επιλέγει \en " \gr Σύγκριση Αυτοκινήτων \en " \gr 
		\item Το σύστημα προτρέπει τον χρήστη να εισάγει τους κωδικούς των αγγελιών, τα οχήματα των οποίων επιθυμεί να συγκρίνει
		\item Ο χρήστης εισάγει τους κωδικούς και καθορίζει τα κριτήρια σύγκρισης των οχημάτων
		\item Το σύστημα προτρέπει τον χρήστη να εισάγει το επιθυμητό εύρος τιμών
		\item Ο χρήστης εισάγει το επιθυμητό εύρος τιμών
		\item Το σύστημα εμφανίζει μια λίστα με τα αυτοκίνητα και τις ιδιότητες σύγκρισης αλλά και το κόστος των τελών κυκλοφορίας και των ασφαλίστρων
		\item Ο χρήστης ταξινομεί την λίστα ως προς τα κριτήρια που είναι τα πιο σημαντικά για αυτόν
		\item Το σύστημα επιλέγει από τα αυτοκίνητα της λίστας και προτείνει στον χρήστη κατάλληλα οχήματα με βάση τα κριτήρια σύγκρισης
		\item Ο χρήστης επιλέγει όχημα 
		\item Το σύστημα μεταφέρει τον χρήστη στην αγγελία του επιλεγμένου οχήματος
	\end{enumerate}
	
	\paragraph{Εναλλακτική Ροή 1}
	
	\begin{enumerate}
		\item Ο χρήστης δεν εισάγει κωδικούς αγγελιών
		\item Το σύστημα προτείνει στον χρήστη οχήματα με βάση τα αυτοκίνητα που έχει αποθηκεύσει στην \en wishlist \gr του αλλά και οχήματα που συμμετέχουν συχνά σε συγκρίσεις άλλων χρηστών
		\item Ο χρήστης καθορίζει τα κριτήρια σύγκρισης και η Περίπτωση Χρήσης συνεχίζει από το βήμα 4 της βασικής ροής
	\end{enumerate}

	\paragraph{\en Use Case 6: \gr Προσθήκη Καταστήματος Αντιπροσωπείας}
	
	\begin{enumerate}
		\item Ο υπεύθυνος της αντιπροσωπείας επιλέγει \en " \gr Προσθήκη Καταστήματος \en " \gr
		\item Το σύστημα ζητά από τον χρήστη το όνομα της εταιρείας στην οποία υπάγεται η αντιπροσωπεία
		\item Ο υπεύθυνος εισάγει το όνομα της εταιρείας
		\item Το σύστημα επιβεβαιώνει πως στην Βάση Δεδομένων της πλατφόρμας, υπάρχει εγγεγραμμένη η αντίστοιχη εταιρεία		
		\item Το σύστημα εμφανίζει τον χάρτη και ζητά από τον χρήστη να εισάγει την τοποθεσία του καταστήματος
		\item Ο υπεύθυνος της αντιπροσωπείας εισάγει τα λεπτομερή γεωγραφικά στοιχεία του καταστήματος
		\item Το σύστημα εντοπίζει το κατάστημα στον χάρτη και ζητά επιβεβαίωση από τον χρήστη
		\item Ο υπεύθυνος επιβεβαιώνει την ορθότητα των στοιχείων		
		\item Το σύστημα ζητά από τον χρήστη να εισάγει τον τίτλο του καταστήματος και μια λίστα με τα αυτοκίνητα που διαθέτει προς πώληση
		\item Ο υπεύθυνος εισάγει τον τίτλο και τα οχήματα που διαθέτει το κατάστημα
		\item Το σύστημα ρωτάει τον χρήστη αν επιθυμεί να δημιουργήσει μια διαφήμιση για το συγκεκριμένο κατάστημα, με σκοπό την ενημέρωση των χρηστών της πλατφόρμας που βρίσκονται στην περιοχή του καταστήματος
		\item Ο υπεύθυνος επιλέγει την δημιουργία της σχετικής διαφήμισης
		\item Το σύστημα ανακατευθύνει τον χρήστη στο παράθυρο δημιουργίας διαφήμισης			
	\end{enumerate}

	\paragraph{Εναλλακτική Ροή 1}
	
	\begin{enumerate}
		\item Ο υπεύθυνος της αντιπροσωπείας εισάγει όνομα εταιρείας, η οποία δεν ανήκει στην πλατφόρμα
		\item Το σύστημα ενημερώνει τον χρήστη σχετικά με το σφάλμα και τον ρωτά αν επιθυμεί να εγγράψει στην πλατφόρμα, την εταιρεία με το όνομα που εισήγαγε
		\item Ο υπεύθυνος επιλέγει εγγραφή 
		\item Το σύστημα ανακατευθύνει τον χρήστη στο μενού εγγραφής εταιρείας
	\end{enumerate}

	\paragraph{\en Use Case 7: \gr Προγραμματισμός \en Test Drive \gr}
	
	\begin{enumerate}
		\item Ο χρήστης επιλέγει \en "Test Drive" \gr
		\item Το σύστημα προτρέπει τον χρήστη να εισάγει τον αριθμό της αγγελίας του οχήματος
		\item Ο χρήστης εισάγει τον αριθμό της αγγελίας
		\item Το σύστημα ζητά από τον χρήστη να επιλέξει την ημερομηνία και ώρα που επιθυμεί 
		\item Ο χρήστης επιλέγει ημερομηνία και ώρα
		\item Το σύστημα εμφανίζει μήνυμα επιτυχούς προγραμματισμού \en Test Drive \gr και αποστέλλει στο \en email \gr του χρήστη και του πωλητή του οχήματος, τα λεπτομερή στοιχεία του ραντεβού 
	\end{enumerate}

	\paragraph{Εναλλακτική Ροή}
	
	\begin{enumerate}
		\item Ο χρήστης επιλέγει μη-διαθέσιμη ημερομηνία και ώρα
		\item Το σύστημα εμφανίζει στο σχετικό μήνυμα σφάλματος και προτρέπει τον χρήστη να επιλέξει ξανά
		\item Ο χρήστης επιλέγει νέα ημερομηνία και ώρα και η Περίπτωση Χρήσης συνεχίζει από το βήμα 6 της βασικής ροής
	\end{enumerate}


    \paragraph{\en Use Case 8: \gr  Ανταλλαγή Οχημάτων \gr}
    
    \begin{enumerate}
    	\item Ο χρήστης επιλέγει \en"\gr Αγορά οχήματος με ανταλλαγή\en" \gr 
    	\item Το σύστημα ζητά από τον χρήστη να εισάγει τον κωδικό της αγγελίας, το όχημα της οποίας θα λάβει μέρος στην ανταλλαγή
    	\item Ο χρήστης εισάγει τον κωδικό της αγγελίας
    	\item Το σύστημα ζητά από τον χρήστη να αναφέρει το είδος, τη μάρκα, το μοντέλο,την κατάσταση, τα χλμ και τυχόν μηχανικά προβλήματα του οχήματος που κατέχει και προτίθεται να ανταλλάξει
    	\item Ο χρήστης αναφέρει λεπτομερώς τα στοιχεία του οχήματος
    	\item Το σύστημα εμφανίζει στον χρήστη μια εκτίμηση της τιμής του οχήματος
    	\item Ο χρήστης αποδέχεται την προτεινόμενη τιμή
    	\item To σύστημα προτρέπει τον χρήστη να ανεβάσει φωτογραφίες του οχήματος του
    	\item O χρήστης αναρτά φωτογραφίες του οχήματος
    	\item Το σύστημα αποστέλλει \en email \gr στον πωλητή του οχήματος της αγγελίας, προκειμένου να τον ενημερώσει για την ανταλλαγή
    	\item O πωλητής μέσω των μηνυμάτων της εφαρμογής στέλνει μια προτεινόμενη τιμή για το όχημα στον χρήστη
    	\item Οι δύο χρήστες ανεβάζουν στην πλατφόρμα τα απαραίτητα νομικά έγγραφα για την μεταβίβαση των οχημάτων και η ανταλλαγή ολοκληρώνεται
    \end{enumerate}
    
    \paragraph{Εναλλακτική Ροή 1}
    
    \begin{enumerate}
    	\item Ο χρήστης δεν αποδέχεται την προτεινόμενη από το σύστημα, ενδεικτική τιμή για το όχημά του
    	\item Το σύστημα ζητά από τον χρήστη να ορίσει τιμή
    	\item O χρήστης επιλέγει μία τιμή και η διαδικασία προχωρά από το βήμα 8 της βασικής ροής
    \end{enumerate}
    
    \paragraph{Εναλλακτική Ροή 2}
    
    \begin{enumerate}
    	\item Ο χρήστης δεν αναρτά φωτογραφίες για το όχημα του
    	\item Το σύστημα εμφανίζει κατάλληλο μήνυμα σφάλματος και αναφέρει στον χρήστη πως η ανάρτηση φωτογραφιών είναι απαραίτητη προϋπόθεση για την συνέχεια της διαδικασίας ανταλλαγής 
    	\item O χρήστης ανεβάζει φωτογραφίες για το όχημα του και η διαδικασία προχωρά από το βήμα 10 της βασικής ροής
    \end{enumerate}
    
    \paragraph{Εναλλακτική Ροή 3}
    \begin{enumerate}
    	\item Το σύστημα εμφανίζει μήνυμα σφάλματος πως οι τιμές των δύο οχημάτων διαφέρουν σημαντικά 
    	\item Η ανταλλαγή ακυρώνεται και ο χρήστης επιστρέφει στην αρχική οθόνη
    \end{enumerate}

	\paragraph{\en Use Case 9: \gr Αγορά Οχήματος\gr}
	
	\begin{enumerate}
		\item Ο χρήστης επιλέγει \en " \gr Αγορά Οχήματος \en " \gr
		\item Το σύστημα αποστέλλει έναν κωδικό ασφαλείας στο \en email \gr του χρήστη
		\item Ο χρήστης εισάγει τον κωδικό ασφαλείας		
		\item Το σύστημα ζητά από τον χρήστη τον κωδικό της αγγελίας του οχήματος που επιθυμεί να αγοράσει
		\item Ο χρήστης εισάγει τον κωδικό της αγγελίας	
		\item To σύστημα ρωτά τον χρήστη αν επιθυμεί να πληρώσει με άτοκες δόσεις
		\item Ο χρήστης αποκρίνεται καταφατικά			
		\item Το σύστημα ρωτά τον χρήστη αν επιθυμεί να χρησιμοποιήσει την υπηρεσία \en " \gr Οικονομικός Σύμβουλος \en " \gr.
		\item Ο χρήστης επιλέγει να χρησιμοποιήσει την υπηρεσία προκειμένου να διαπιστώσει αν μπορεί να αντεπεξέλθει στα έξοδα 
		\item Το σύστημα ζητά από τον χρήστη να εισάγει τον μηνιαίο μισθό του, με σκοπό τον υπολογισμό ενός προσαρμοσμένου στον χρήστη, ποσού άτοκης μηνιαίας δόσης
		\item Το σύστημα εμφανίζει στον χρήστη το υπολογισμένο ποσό καθώς και το κόστος των τελών κυκλοφορίας του οχήματος
		\item Ο χρήστης αποδέχεται το ποσό της μηνιαίας δόσης
		\item Το σύστημα εμφανίζει την συνολική τιμή και μεταφέρει τον χρήστη στην σελίδα του συστήματος πληρωμών
		\item Ο χρήστης πληρώνει για την αγορά του οχήματος
		\item Το σύστημα εμφανίζει μήνυμα επιτυχούς αγοράς και αποστέλλει στο \en email \gr του χρήστη την απόδειξη πληρωμής καθώς και τον κωδικό της συναλλαγής
	\end{enumerate}

	\paragraph{Εναλλακτική Ροή 1}
	\begin{enumerate}
		\item Ο χρήστης επιλέγει να μην πληρώσει με άτοκες δόσεις και η Περίπτωση Χρήσης συνεχίζει από το βήμα 13 της βασικής ροής
	\end{enumerate}

	\paragraph{Εναλλακτική Ροή 2}
	\begin{enumerate}
		\item Ο χρήστης εισάγει κωδικό μη-υπαρκτής αγγελίας
		\item Το σύστημα ενημερώνει τον χρήστη
		\item Ο χρήστης διορθώνει τον κωδικό και η Περίπτωση Χρήσης συνεχίζει από το βήμα 6 της βασικής ροής
	\end{enumerate}


\end{document}
 
