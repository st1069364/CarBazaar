%% Overleaf			
%% Software Manual and Technical Document Template	
%% 									
%% This provides an example of a software manual created in Overleaf.

\documentclass{../ol-softwaremanual}

% Packages used in this example
\usepackage{graphicx}  % for including images
\usepackage{microtype} % for typographical enhancements
\usepackage{minted}    % for code listings
\usepackage{amsmath}   % for equations and mathematics
\setminted{style=friendly,fontsize=\small}
\renewcommand{\listoflistingscaption}{List of Code Listings}
\usepackage{hyperref}  % for hyperlinks
\usepackage[a4paper,top=4.2cm,bottom=4.2cm,left=3.5cm,right=3.5cm]{geometry} % for setting page size and margins

\usepackage[english, greek]{babel}

\usepackage{subfig}




% Custom macros used in this example document
\newcommand{\doclink}[2]{\href{#1}{#2}\footnote{\url{#1}}}
\newcommand{\cs}[1]{\texttt{\textbackslash #1}}

\begin{document}
	
	
	\begin{titlepage}
		
		
		% Frontmatter data; appears on title page
		\title{\en Project Code \\}
		\version{1.0}
		\softwarelogo{\includegraphics[scale=0.4]{../CarBazaar_logo.png}}
	\end{titlepage}
	
	
	\maketitle
	
	\newpage
	
	\center{\textbf{Μέλη Ομάδας}}
	
	\vspace{20pt}
	
	
	
	\begin{table}[htbp!]
		
		\begin{tabular}{llll}
			Μεμελετζόγλου Χαρίλαος & 1069364 & \en st1069364@ceid.upatras.gr & 4o Έτος   \\ 
			\\ Λέκκας Γεώργιος      &      1067430    &   \en st1067430@ceid.upatras.gr & 4o Έτος  \\
			\\ Γιαννουλάκης Ανδρέας        &   1067387       & \en st1067387@ceid.upatras.gr & 4o Έτος           \\
			\\ Κανελλόπουλος Ιωακείμ        &  1070914        &    \en st1070914@ceid.upatras.gr & 4o Έτος        \\ 
		\end{tabular}
	\end{table}
	
	\center{\textbf{Υπεύθυνοι Παρόντος Τεχνικού Κειμένου}}
	
	\vspace{20pt}
	
	\begin{table}[htbp!]
		\begin{tabular}{ll}
			Μεμελετζόγλου Χαρίλαος & \en Editor \\
			\\ Λέκκας Γεώργιος      &   \en  Peer Reviewer \\
			\\ Γιαννουλάκης Ανδρέας & \en Peer Reviewer \\
			\\ Κανελλόπουλος Ιωακείμ & \en Peer Reviewer \\ 
		\end{tabular}
	\end{table}
	
	\vspace{10pt}
	\center{\textbf{Αλλαγές στην έκδοση \en v1.0 \gr}}
	\vspace{10pt}
	\flushleft
	
	Αφαιρέθηκε η παράγραφος για τον \en driver \gr κώδικα του \en use case "\gr Ανάρτηση Αγγελίας Πώλησης Οχήματος\en"\gr. Προσθήκη της παραγράφου που αφορά το αρχείο \en demo.py \gr, που περιέχει το \en demo \gr της εφαρμογής αλλά και της παραγράφου για το αρχείο \en test\_data\_init.py\gr. \break
	
	Τέλος, προστέθηκε η παράγραφος για το \en README \gr αρχείο που βρίσκεται στο \en \textit{src} directory \gr.

	
	
	
	\newpage
	
	\center{\textbf{Εργαλεία που χρησιμοποιήθηκαν}}
	
	\vspace{20pt}
	\flushleft
	Χρησιμοποιήθηκε το \en \doclink{https://www.overleaf.com/}{Overleaf} \gr και το \en \doclink{https://www.texstudio.org/}{TexStudio} \gr για την συγγραφή του \LaTeX\ κώδικα. \break
	
	Για την δημιουργία του \en UI \gr, χρησιμοποιείται το \en \textbf{PyQt} cross-platform Qt Framework \gr. \break
	
	Για την συγγραφή του \en Python \gr κώδικα, χρησιμοποιείται το \en IDE \doclink{https://www.jetbrains.com/pycharm/}{PyCharm} \gr και το \en \doclink{https://code.visualstudio.com/}{VSCode} \gr .  \break 
	
	Για την αποδοτικότερη δημιουργία των οθονών, χρησιμοποιείται το \en \doclink{https://doc.qt.io/qt-5/qtdesigner-manual.html}{QtDesigner} \gr, το οποίο παράγει τα \en .ui \gr, αρχεία που βρίσκονται στο \en src/ui/qt\_ui directory \gr . \break
	
	
	\newpage
	
	
	\center{\textbf{Περιγραφή κώδικα στο \en Github \gr}}
	
	\flushleft
	
	Η ομάδα μας έχει χρησιμοποιήσει το \en \doclink{https://github.com/st1069364/CarBazaar}{Github} \gr από την αρχή του \en Project \gr, ακόμα και για την δημιουργία των τεχνικών κειμένων. Ως εκ τούτου, στο \en Repo \gr υπάρχουν και τεχνικά κείμενα αλλά και κώδικας. \break
	
	
	Ο κώδικας βρίσκεται στο \en \doclink{https://github.com/st1069364/CarBazaar/tree/main/src}{src} directory \gr, ενώ τα τεχνικά κείμενα βρίσκονται στο \en \doclink{https://github.com/st1069364/CarBazaar/tree/main/doc}{doc} directory \gr .\break
	
	Συγκεκριμένα, στο \en \textbf{src} directory \gr, υπάρχει ο κώδικας του \en project \gr. \break
	
	Στο αρχείο \en \textbf{src/classes.py} \gr, υπάρχουν οι κλάσεις του \en Class Diagram \gr, καθώς και οι μέθοδοί τους. Επίσης, υπάρχουν και μέθοδοι που δεν εμφανίζονται στο \en Class Diagram \gr, (πχ \en getters/setters \gr), αλλά υλοποιούνται καθώς (λογικά) θα χρειαζόντουσαν στην πλήρη έκδοση της εφαρμογής. \break	
	
	\red{Στο αρχείο \en\textbf{test\_data\_init.py}\gr, ορίζεται μια συνάρτηση η οποία δημιουργεί και αρχικοποιεί κάποια \en test data \gr (όπως \en instances \gr αυτοκινήτων, χρηστών, αγγελιών), τα οποία χρειάζονται για την ομαλή εκτέλεση του \en demo \gr.} \break
	
	\red{Στο αρχείο \en \textbf{demo.py} \gr, υπάρχει ένα \en demo \gr της εφαρμογής, όπου έχει υλοποιηθεί πλήρως το \en GUI \gr, των \en use cases "\gr Ανάρτηση Αγγελίας Πώλησης Οχήματος\en" \gr και \en"\gr Προγραμματισμός Ελέγχου Οχήματος\en"\gr.} \break

	Στις οθόνες, έχει ενσωματωθεί διαδραστικός χάρτης (μέσω του \en OpenStreetMaps API \gr), ο οποίος έχει προκύψει μέσω του \en QML \gr αρχείου \en \textit{src/ui/qml/map.qml}\gr. \break	
	
	Στο \en directory \textbf{ui/qt\_ui} \gr υπάρχουν τα απαραίτητα \en resources \gr, καθώς και τα \en .ui \gr αρχεία που προκύπτουν από τον \en Qt Designer \gr. \break	
	
	Το \en \textit{ui/app\_res\_rc.py} \gr αρχείο περιέχει την \en compiled \gr έκδοση του \en .qrc \gr αρχείου που ομαδοποιεί τα \en resources \gr του \en Qt Designer \gr. \break
	
	
	Στο \en \textbf{doc} directory \gr, υπάρχουν όλα τα Τεχνικά Κείμενα. \break
	
	Χρησιμοποιούμε τα \en \textit{Releases} \gr που προσφέρει το \en Github \gr, ώστε να ομαδοποιούμε τα τεχνικά κείμενα του κάθε παραδοτέου σε ένα \en release \gr .\break
	
	\red{Τέλος, στο \en \textbf{src} directory \gr, υπάρχει ένα \en README \gr αρχείο το οποίο περιγράφει τα βήματα που πρέπει να ακολουθηθούν ώστε να μπορέσουν να εγκατασταθούν τα απαραίτητα πακέτα και να εκτελεστεί το \en demo \gr. Επίσης, στο αρχείο αυτό υπάρχει ενσωματωμένο ένα \en gif \gr, το οποίο επιδεικνύει την εκτέλεση του \en demo \gr.}
	
	
	
	
	
	
	
	
	
	
\end{document}
