%% Overleaf			
%% Software Manual and Technical Document Template	
%% 									
%% This provides an example of a software manual created in Overleaf.

\documentclass{../ol-softwaremanual}

% Packages used in this example
\usepackage{graphicx}  % for including images
\usepackage{microtype} % for typographical enhancements
\usepackage{minted}    % for code listings
\usepackage{amsmath}   % for equations and mathematics
\setminted{style=friendly,fontsize=\small}
\renewcommand{\listoflistingscaption}{List of Code Listings}
\usepackage{hyperref}  % for hyperlinks
\usepackage[a4paper,top=4.2cm,bottom=4.2cm,left=3.5cm,right=3.5cm]{geometry} % for setting page size and margins

\usepackage[english, greek]{babel}

\usepackage{subfig}

\usepackage{incgraph,tikz}

\usepackage{filemod}
\usepackage{xcolor}




\usepackage{rotating}


% Custom macros used in this example document
\newcommand{\doclink}[2]{\href{#1}{#2}\footnote{\url{#1}}}
\newcommand{\cs}[1]{\texttt{\textbackslash #1}}

\begin{document}
	
	
	\begin{titlepage}
		
		
		% Frontmatter data; appears on title page
		\title{\en Use Cases \\}
		\version{0.4}
		\softwarelogo{\includegraphics[scale=0.4]{../CarBazaar_logo.png}}		
		
	\end{titlepage}
	
	
	\maketitle
	
	\newpage
	
	\center{\textbf{Μέλη Ομάδας}}
	
	\vspace{20pt}
	
	
	
	\begin{table}[htbp!]
		
		\begin{tabular}{llll}
			Μεμελετζόγλου Χαρίλαος & 1069364 & \en st1069364@ceid.upatras.gr & 4o Έτος   \\ 
			\\ Λέκκας Γεώργιος      &      1067430    &   \en st1067430@ceid.upatras.gr & 4o Έτος  \\
			\\ Γιαννουλάκης Ανδρέας        &   1067387       & \en st1067387@ceid.upatras.gr & 4o Έτος           \\
			\\ Κανελλόπουλος Ιωακείμ        &  1070914        &    \en st1070914@ceid.upatras.gr & 4o Έτος        \\ 
		\end{tabular}
	\end{table}
	
	\center{\textbf{Υπεύθυνοι Παρόντος Τεχνικού Κειμένου}}
	
	\vspace{20pt}
	
	\begin{table}[htbp!]
		\begin{tabular}{ll}
			Μεμελετζόγλου Χαρίλαος & \en Editor \\
			\\ Λέκκας Γεώργιος      &   \en  Editor \\
			\\ Γιαννουλάκης Ανδρέας & \en Contributor \\
			\\ Κανελλόπουλος Ιωακείμ & \en Contributor \\ 
		\end{tabular}
	\end{table}
	
	
	\center{\textbf{Αλλαγές στην έκδοση \en v0.4 \gr}}
	
	\flushleft
	
	\begin{itemize}
		\item \en Use Case \textbf{1} \gr (\textit{Ανάρτηση Αγγελίας Πώλησης Οχήματος}) : Κάποιες από τις  αλλαγές που έγιναν αφορούν την σειρά εμφάνισης των οθονών. Συγκεκριμένα, η πρώτη οθόνη που εμφανίζεται είναι η οθόνη που αφορά τα Στοιχεία του Οχήματος, και η εισαγωγή της τοποθεσίας του χρήστη και του τίτλου της αγγελίας, μετακινήθηκε στο βήμα \textbf{10}. Η αλλαγή αυτή έγινε ώστε η εισαγωγή των στοιχείων αυτών να βρίσκεται λίγο πριν την δημιουργία της Οντότητας της αγγελίας (\en \textit{CarListing}\gr), ώστε να είναι ευκολότερη η συγγραφή του κώδικα. \\
		
		Επίσης, πλέον στην οθόνη που γίνεται η ανάρτηση των εγγράφων πιστοποίησης κατάστασης του οχήματος, γίνεται και η ανάρτηση των φωτογραφιών (έγινε και μετονομασία της οθόνης αυτής).
		
		Ακόμη, μεταξύ λοιπών αλλαγών, προστέθηκε νέο βήμα \textbf{2}, όπου όταν εμφανιστεί η οθόνη Στοιχείων Οχήματος, δημιουργείται και η Οντότητα του Αυτοκινήτου, που θα χρησιμοποιηθεί για την προσωρινή αποθήκευση των στοιχείων του οχήματος, ώστε να μπορέσει να γίνει ο έλεγχος ύπαρξης του οχήματος.
		
		Στο βήμα \textbf{4} γίνεται προσθήκη των στοιχείων στην Οντότητα του Αυτοκινήτου και ο παραπάνω έλεγχος.
		
		Στην συνέχεια, στο βήμα \textbf{5} γίνεται δημιουργία της Αγγελίας του Οχήματος και προσθήκη σε αυτήν, του οχήματος.
		
		Προσθήκη της τιμής του οχήματος στην Αγγελία, στο βήμα \textbf{10}.	
		
		Αλλαγή της συνθήκης ενεργοποίησης της Εναλλακτικής Ροής \en \# 2\gr, από \textit{μη-ανάρτηση φωτογραφιών ή μη-προσθήκη περιγραφής} σε \textit{μη-προσθήκη περιγραφής ή μη-προσθήκη τίτλου αγγελίας}.
		
		\item \en Use Case \textbf{2} \gr (\textit{Προγραμματισμός Ελέγχου Οχήματος}) : Αλλαγή του τίτλου της εναρκτήριας ενέργειας του χρήστη από \textit{Έλεγχος Οχήματος} σε \textit{Προγραμματισμός Ελέγχου Οχήματος}.
		
		Τροποποίηση του βήματος \textbf{4} της Βασικής Ροής. Πλέον η δημιουργία της Οντότητας του Ελέγχου (\en \textit{CarInspection}\gr), γίνεται σε αυτό το βήμα.
		
		Η εμφάνιση της οθόνης των στοιχείων του Ελεγκτή, μεταφέρθηκε στο βήμα \textbf{5}.
		
		Τροποποίηση του βήματος \textbf{6}, σχετικά με την επιλογή του χρήστη να συνεχίσει ή όχι με τον προτεινόμενο Ελεγκτή.
		
		Αλλαγή διατύπωσης της ανάκτησης της αγγελίας στο βήμα \textbf{10}.
		
		Νέο βήμα \textbf{12}, όπου μετά την επιτυχή πληρωμή, δημιουργείται η Οντότητα της Απόδειξης, και προστίθενται σε αυτήν τα απαραίτητα στοιχεία.
		
		Νέο βήμα \textbf{13}, όπου γίνεται η εισαγωγή των στοιχείων του Ελέγχου στην αντίστοιχη Οντότητα, καταχώρηση του Ελέγχου στην λίστα των εκκρεμών Ελέγχων, του επιλαχόντα Ελεγκτή.
		
		Τέλος, στο βήμα \textbf{14}, γίνεται και ανάκτηση της Απόδειξης της Συναλλαγής, πριν την αποστολή του \en email \gr. Επίσης, η τελική οθόνη που εμφανίζει το σύστημα, μετονομάστηκε από \textit{Οθόνη Επιτυχούς Κράτησης} σε \textit{Οθόνη Επιτυχούς Προγραμματισμού Ελέγχου}	

		Αλλαγή διατύπωσης του βήματος \textbf{4} της Εναλλακτικής Ροής \en \# 2 \gr. Προσθήκη του βήματος \textbf{5}, όπου γίνεται η επιστροφή του χρήστη στην  προηγούμενη οθόνη.
		
		Η περίπτωση όπου τα στοιχεία που εισήγαγε ο χρήστης, δεν αντιστοιχούν σε έγκυρο Ελεγκτή, μεταφέρθηκε στην \textit{Ένθετη Εναλλακτική Ροή 2.1}.
	
		\item \en Use Case \textbf{5} \gr (\textit{Σύγκριση Αυτοκινήτων}) : Το βήμα \textbf{5} εμπλουτίστηκε με την προσθήκη της λίστας των αγγελιών στην Οντότητα της Σύγκρισης. Αλλαγή διατύπωσης του βήματος \textbf{2} της Εναλλακτικής Ροής \en \# 1 \gr.
		
				
		\item \en Use Case \textbf{6} \gr (\textit{Προσθήκη Καταστήματος Αντιπροσωπείας}) : Μεταφορά στο βήμα \textbf{7}, της δημιουργίας της Οντότητας του καταστήματος, της προσθήκης σε αυτήν των απαραίτητων στοιχείων και του ελέγχου ύπαρξης του καταστήματος. 
		
		Μεταφορά της εμφάνισης της οθόνης Λεπτομερειών Καταστήματος, στο βήμα \textbf{8}.
		
		Νέο βήμα \textbf{12}, όπου καταχωρείται το κατάστημα, καθώς η δημιουργία έχει εκτελεστεί στο νέο βήμα \textbf{7}.
		
		\item \en Use Case \textbf{7} \gr (\textit{Προγραμματισμός \en Test Drive \gr}) : Η δημιουργία της Οντότητας του \en Test Drive \gr, μεταφέρθηκε στο βήμα \textbf{4}, από το τελευταίο βήμα όπου βρισκόταν μέχρι πρότινος. Η εμφάνιση της οθόνης Προγραμματισμού \en Test Drive \gr, μετακινήθηκε από το βήμα 4, στο βήμα \textbf{5}. Επίσης, στο βήμα \textbf{4} γίνεται η προσθήκη της αγγελίας του οχήματος και του ενδιαφερόμενου οδηγού, στην οντότητα του \en Test Drive \gr.
		
		Στο βήμα \textbf{7}, το σύστημα προσθέτει στην Οντότητα του \en Test Drive \gr, την επιλεγμένη ημερομηνία και ελέγχει την διαθεσιμότητα της ημερομηνίας.
		
		Τέλος, στο τελευταίο βήμα γίνεται πλέον μόνο \textit{καταχώρηση} του \en Test Drive \gr και όχι και δημιουργία του.
		
		\item \en Use Case \textbf{8} \gr (\textit{Ανταλλαγή Οχήματος}) : Προσθήκη της δημιουργίας της Οντότητας του Οχήματος, στο βήμα \textbf{2} της Βασικής Ροής. Νέο βήμα \textbf{4} που αφορά την προσθήκη των στοιχείων στην Οντότητα του Αυτοκινήτου και του ελέγχου ύπαρξης του οχήματος. 
		
		Το βήμα \textbf{5} αποτελεί μέρος του πρώην βήματος 4.
		
		Προσθήκη του βήματος \textbf{8}, όπου γίνεται ανάκτηση των στοιχείων του επιλεγμένου καταστήματος και προσθήκη του, ως επιλεγμένου καταστήματος, στην Οντότητα της Ανταλλαγής. Το βήμα \textbf{9} είναι το πρώην βήμα 7.
		
		Προσθήκη του βήματος \textbf{13}, όπου μετά την αποδοχή της Ανταλλαγής από τον χρήστη, δημιουργείται η αντίστοιχη Συναλλαγή και γίνεται προσθήκη αυτής στην Οντότητα της Ανταλλαγής.
		
		Προσθήκη της  ανάκτησης των νομικών εγγράφων της Ανταλλαγής, στις ενέργειες του συστήματος στο τελευταίο βήμα.
		
		\item \en Use Case \textbf{9} \gr (\textit{Αγορά Οχήματος}) : Τροποποίηση του βήματος \textbf{6}, ώστε να περιλαμβάνει την ανάκτηση της Αγγελίας του Οχήματος. Αλλαγή διατύπωσης στο βήμα \textbf{7}, όπου ο χρήστης επιλέγει τον τρόπο πληρωμής
		
		Προσθήκη του ελέγχου σχετικά με τον τρόπο πληρωμής, στο βήμα \textbf{8}.
		
		Προσθήκη του βήματος \textbf{9}, όπου δημιουργείται η Οντότητα της Μηνιαίας Δόσης (\en  \textit{MonthlyInstallment} \gr), η ανάκτηση από την Αγγελία της τιμής του οχήματος, και η προσθήκη αυτής στην Οντοτητα της Μηνιαίας Δόσης.
		
		Νέο βήμα \textbf{11}, όπου αναδεικνύεται ρητά ο \textit{υπολογισμός} του ποσού της δόσης, από το σύστημα.
		
		Προστέθηκε το βήμα \textbf{13}, όπου το σύστημα επιστρέφει τον χρήστη στην οθόνη Διαχείρισης Οικονομικών, ώστε να είναι δυνατόν να συνεχίσει το \en use case \gr, από την ίδια οθόνη ακόμα και στην περίπτωση που ο χρήστης επέλεξε να μην πληρώσει με δόσεις.
		
		Το βήμα \textbf{14}, δεν ενσωματώθηκε στο βήμα \textbf{13}, ώστε να είναι δυνατή η επιστροφή από την Εναλλακτική Ροή \en \# 1 \gr, πίσω στην Βασική Ροή, στο βήμα 14, μιας και η επιστροφή στο βήμα 13 δεν θα είχε νόημα καθώς ο χρήστης θα βρισκόταν ήδη στην οθόνη Διαχείρισης Οικονομικών. Επίσης, έγινε αλλαγή διατύπωσης της εναρκτήριας συνθήκης την προαναφερθείσας Εναλλακτικής Ροής.
		
		Το βήμα \textbf{15} είναι το πρώην βήμα 13.
		
		Προσθήκη του βήματος \textbf{16}, όπου ο χρήστης επιβεβαιώνει την ορθότητα των στοιχείων της αγοράς, πριν γίνει μετάβαση στο μενού πληρωμών.
		
		Προσθήκη του βήματος \textbf{17}, όπου μετά την επιτυχή συναλλαγή, δημιουργείται η Οντότητα της Απόδειξης, και προστίθενται σε αυτήν τα απαραίτητα στοιχεία, αλλά και προστίθεται η Συναλλαγή στην Οντότητα της Μηνιαίας Δόσης.
		
		Τέλος, στις τελευταίες ενέργειες του συστήματος, μετά την καταχώρηση στο \en \textit{TransactionLog} \gr, γίνεται προσθήκη του οχήματος στην λίστα των αυτοκινήτων του Χρήστη αλλά και ανάκτηση της Απόδειξης της Συναλλαγής.
		
		\item \en Use Case \textbf{10} \gr (\textit{Αναζήτηση Οχήματος}) : Τροποποίηση του βήματος \textbf{6} της Βασικής Ροής. Πλέον, μετά τον έλεγχο του συστήματος για την συμπλήρωση των πεδίων της αναζήτησης, γίνεται προσθήκη των απαραίτητων  πληροφοριών στην Οντότητα της Αναζήτησης Οχήματος και εντοπισμός του χρήστη.
		
		Αφαίρεση της προσθήκης πληροφοριών στην Οντότητα της Αναζήτησης, από το βήμα \textbf{7}. Ακόμη, αφαιρέθηκε ο προσδιορισμός \textit{Εφήμερο Αντικείμενο} από τα Αποτελέσματα της Αναζήτησης, μιας και πλέον αποτελεί \en attribute \gr, της κλάσης \en\textit{CarSearch} \gr, και όχι εφήμερο αντικείμενο.
		
		
		\item \en Use Case \textbf{11} \gr (\textit{Ανάρτηση Αγγελίας Πώλησης Ανταλλακτικού}) : Η δημιουργία της Οντότητα της Αγγελίας του Ανταλλακτικού (\en \textit{SparePartListing}\gr), μεταφέρθηκε στο βήμα \textbf{4}, καθώς και η προσθήκη κάποιων πληροφοριών σε αυτήν.
		
		Η δημιουργία της Οντότητας του Ανταλλακτικού (\en \textit{SparePart}\gr), μεταφέρθηκε στο βήμα \textbf{5} της Βασικής Ροής. Πλέον η προσθήκη των πληροφοριών στην προαναφερθείσα Οντότητα, γίνεται στο βήμα \textbf{7}, \textit{πριν} τον έλεγχο εγκυρότητας του κωδικού του Ανταλλακτικού.
		
		Επίσης, τροποποίηση του βήματος \textbf{8}, ώστε έπειτα από τον παραπάνω έλεγχο, να γίνεται εισαγωγή του ανταλλακτικού στην κατάλληλη κατηγορία, και προσθήκη στην Αγγελία των απαραίτητων πληροφοριών.
		
		Στο βήμα \textbf{11}, γίνεται πλέον μόνο προσθήκη της περιγραφής και των φωτογραφιών στην Οντότητα της Αγγελίας, καθώς η δημιουργία της έχει πραγματοποιηθεί σε προηγούμενο βήμα.
		
		
		\item \en Use Case \textbf{12} \gr (\textit{Έλεγχος Αναφοράς Αγγελίας}) : Προσθήκη του βήματος \textbf{9}, όπου δημιουργείται η Οντότητα της Φόρμας Διαγραφής της Αγγελίας. Προσθήκη της ανάκτησης της Φόρμας Διαγραφής, στο βήμα \textbf{10} της Βασικής Ροής.
		
		Επίσης, το \en Use Case \gr, μετονομάστηκε σε \textbf{\textit{Έλεγχος Αναφοράς Αγγελίας}} από \textit{Έλεγχος Αναφοράς}
		
		\item \en Use Case \textbf{13} \gr (\textit{Αγορά Ασφαλιστικού Πακέτου}) : Αναδιατύπωση του βήματος\textbf{6} που αφορά την εξαργύρωση των πόντων που εισήγαγε ο χρήστης. Αφαιρέθηκε η ανάκτηση των πόντων, καθώς αυτό γίνεται εσωτερικά στην μέθοδο \en \textit{redeem\_points()} \gr. 
		
		Επίσης, η δημιουργία της Οντότητας του Ασφαλιστικού Πακέτου μεταφέρθηκε στο βήμα \textbf{7}. Στο βήμα αυτό γίνεται και ο υπολογισμό της τιμής των ασφαλίστρων καθώς και η προσθήκη των απαραίτητων πληροφοριών στην προαναφερθείσα Οντότητα. 
		
		Ρητή ανάδειξη της δημιουργίας της Οντότητας της Απόδειξης, καθώς και της προσθήκης σε αυτήν των απαραίτητων στοιχείων, στο βήμα \textbf{9}, έπειτα από την επιτυχή συναλλαγή.
		
		Προσθήκη της ανάκτησης της Απόδειξης, στο βήμα \textbf{10}
		
		\item \en Use Case \textbf{14} \gr (\textit{Μεταφορά Οχήματος}) : Νέο βήμα \textbf{8} στο οποίο γίνεται η δημιουργία της Οντότητας της Μεταφοράς Οχήματος, η προσθήκη των απαραίτητων πληροφοριών σε αυτήν αλλά και η αναζήτηση Μεταφορέα με βάση την τοποθεσία του οχήματος.
		
		Νέο βήμα \textbf{11}, όπου αναδεικνύεται ρητά η δημιουργία της Οντότητας της Απόδειξης Συναλλαγής, μετά την επιτυχή συναλλαγή, και η προσθήκη των απαραίτητων στοιχείων σε αυτήν.
		
		Επίσης, πλέον, στο τελευταίο βήμα \textit{πριν} την καταχώρηση της Συναλλαγής στο \en \textit{TransactionLog} \gr, γίνεται και καταχώρηση της Μεταφοράς Οχήματος αλλά και ανάκτηση της Απόδειξης μετά την καταχώρηση της Συναλλαγής
		
	\end{itemize}
	
	
	\vspace{20pt}
	
	Οι σημαντικές αλλαγές σημειώνονται με \red{κόκκινο} χρώμα.
	
	
	
	
	\newpage 
	\center{\textbf{Εργαλεία που χρησιμοποιήθηκαν}}
	
	\vspace{20pt}
	\flushleft
	Χρησιμοποιήθηκε το \en \doclink{https://www.overleaf.com/}{Overleaf} \gr και το \en \doclink{https://www.texstudio.org/}{TexStudio} \gr για την συγγραφή του \LaTeX\ κώδικα. \break
	
	Για την δημιουργία του λογότυπου, χρησιμοποιήθηκε το εργαλείο \en \doclink{https://www.adobe.com/express/create/logo}{Adobe Express} . \gr \break
	
	Για την δημιουργία του \en UML Use Case Diagram \gr χρησιμοποιήθηκε το \en \doclink{https://www.visual-paradigm.com/}{Visual Paradigm} . \gr \break 
	
	\newpage
	
	\center{\textbf{\en Use Cases \gr}}
	
	\paragraph{\en Use Case 1: \gr Ανάρτηση Αγγελίας Πώλησης Οχήματος}
	
	\begin{enumerate}		
		\item Ο χρήστης επιλέγει \en"\gr Ανάρτηση Αγγελίας Οχήματος\en" \gr στο αρχικό μενού
		\item Το σύστημα εμφανίζει την οθόνη \red{Στοιχεία Οχήματος και δημιουργεί την Οντότητα του οχήματος (\en \textit{Car}\gr)}		
		\red{\item Ο χρήστης εισάγει στοιχεία του οχήματος όπως μάρκα, μοντέλο, έτος κυκλοφορίας, χιλιόμετρα, κυβικά, τύπος καυσίμου, χρώμα, αριθμός πινακίδας, κλπ}		
		\red{\item Το σύστημα προσθέτει στην οντότητα του αυτοκινήτου, τα στοιχεία του οχήματος και ελέγχει πως όντως κυκλοφορεί αντίστοιχο μοντέλο αυτοκινήτου στην αγορά }
		\red{\item Το σύστημα δημιουργεί την Οντότητα της Αγγελίας (\en \textit{CarListing}\gr), προσθέτει σε αυτήν το όχημα και εμφανίζει την Οθόνη Ανάρτησης Αρχείων}		
		\red{\item Ο χρήστης ανεβάζει τα απαραίτητα έγγραφα που έχουν προκύψει από τον έλεγχο του οχήματος, καθώς και τις φωτογραφίες του οχήματος}		
		\item Το σύστημα \red{προσθέτει τα έγγραφα και τις φωτογραφίες στην Αγγελία,} υπολογίζει μια εκτίμηση της τιμής του οχήματος, με βάση την κατάστασή του και εμφανίζει την Οθόνη Τιμής Οχήματος
		\item Ο χρήστης επιλέγει να συνεχίσει με την προτεινόμενη τιμή ή εισάγει δικιά του				
		\item Το σύστημα ελέγχει αν ο χρήστης αποδέχθηκε την προτεινόμενη τιμή. Σε περίπτωση μη-αποδοχής, ελέγχει αν η τιμή που εισήγαγε ο χρήστης παρουσιάζει μεγάλη απόκλιση από την προτεινόμενη τιμή		
		\red{\item Το σύστημα προσθέτει στην Αγγελία την τιμή πώλησης του οχήματος και μεταφέρει τον χρήστη στην οθόνη Εισαγωγής Στοιχείων Αγγελίας, προτρέποντάς τον, να εισάγει την περιγραφή της αγγελίας, τον τίτλο της αλλά και την τοποθεσία του}		
		\red{\item Ο χρήστης προσθέτει το κείμενο της περιγραφής, τον τίτλο της αγγελίας και εισάγει την τοποθεσία του}		
		\red{\item Το σύστημα εντοπίζει τον χρήστη και αφού ελέγξει πως συμπληρώθηκαν τα πεδία περιγραφής και τίτλου, προσθέτει στην Αγγελία τις απαραίτητες πληροφορίες και δημιουργεί το \en 3D \gr μοντέλο του οχήματος }		
		\red{\item Το σύστημα εμφανίζει την Οθόνη Προεπισκόπησης Αγγελίας}
		\item Ο χρήστης εγκρίνει την αγγελία
		\item Το σύστημα καταχωρεί την αγγελία και εμφανίζει μήνυμα επιτυχούς καταχώρησης αγγελίας
	\end{enumerate}
	
	\paragraph{Εναλλακτική Ροή 1}
	
	\begin{enumerate}
		\item O χρήστης εισάγει στοιχεία μη-υπαρκτού μοντέλου
		\item Το σύστημα εμφανίζει προειδοποιητικό μήνυμα, επιστρέφει τον χρήστη στην οθόνη \red{\textit{Στοιχεία Οχήματος}}, προτρέποντάς τον να διορθώσει τα λανθασμένα πεδία
		\item Ο χρήστης προβαίνει στις απαραίτητες διορθώσεις και η Περίπτωση Χρήσης συνεχίζει από το βήμα 4 της βασικής ροής
	\end{enumerate}
	
	\paragraph{Εναλλακτική Ροή 2}
	
	\begin{enumerate}
		\item \red{Ο χρήστης δεν εισάγει περιγραφή ή τίτλο για την αγγελία}
		\item Το σύστημα εμφανίζει προειδοποιητικό μήνυμα, επιστρέφει τον χρήστη στην οθόνη \textit{Εισαγωγή Στοιχείων Αγγελίας} προτρέποντάς τον, να συμπληρώσει τις απαραίτητες πληροφορίες
		\item Ο χρήστης εισάγει τα απαραίτητα δεδομένα και η Περίπτωση Χρήσης συνεχίζει από το βήμα 11 της βασικής ροής
	\end{enumerate}
	
	\paragraph{Εναλλακτική Ροή 3}
	
	\begin{enumerate}
		\item Ο χρήστης εισάγει τιμή η οποία είναι σημαντικά μεγαλύτερη από την προτεινόμενη από το σύστημα, τιμή
		\item Το σύστημα εμφανίζει προειδοποιητικό μήνυμα, επιστρέφει τον χρήστη στον οθόνη \textit{Τιμή Οχήματος}, προτρέποντάς τον, να επιλέξει ξανά σχετικά με την τιμή του οχήματος. Η Περίπτωση Χρήσης συνεχίζει από το βήμα \red{8} της βασικής ροής
	\end{enumerate}
	
	
	\paragraph{\en Use Case 2: \gr Προγραμματισμός Ελέγχου Οχήματος}
	
	\begin{enumerate}
		\item Ο χρήστης επιλέγει \red{\en"\gr Προγραμματισμός Ελέγχου Οχήματος\en" \gr} στο αρχικό μενού
		\item Το σύστημα εμφανίζει την οθόνη Προγραμματισμού Ελέγχου Οχήματος
		\item Ο χρήστης επιλέγει το πακέτο ελέγχου που επιθυμεί, αν επιθυμεί την έκδοση πιστοποιητικών εγγράφων της κατάστασης του οχήματος, την ημερομηνία και ώρα διεξαγωγής του ελέγχου και εισάγει την τοποθεσία του
		\item Το σύστημα αφού επιβεβαιώσει την εισαχθείσα τοποθεσία, \red{δημιουργεί την Οντότητα του Ελέγχου (\en \textit{CarInspection}\gr), προσθέτει σε αυτήν την τοποθεσία που εισήγαγε ο χρήστης και αναζητά έναν προτεινόμενο Ελεγκτή}
		\red{\item Το σύστημα εμφανίζει την οθόνη \en"\gr Στοιχεία Ελεγκτή\en"\gr, προβάλλοντας τα στοιχεία του προτεινόμενου Ελεγκτή}		
		\item Ο χρήστης \red{επιλέγει να συνεχίσει με τον προτεινόμενο Ελεγκτή ή τον απορρίπτει προκειμένου να εισάγει τα στοιχεία του Ελεγκτή της αρεσκείας του}
		\item Το σύστημα ελέγχει αν ο χρήστης αποδέχθηκε τον προτεινόμενο ελεγκτή
		\item Το σύστημα εμφανίζει την οθόνη Εισαγωγή Κωδικού Αγγελίας, στο όχημα της οποίας θα πραγματοποιηθεί ο έλεγχος		
		\item Ο χρήστης εισάγει τον κωδικό της αγγελίας
		\item Το σύστημα \red{ανακτά την αγγελία} και εμφανίζει την οθόνη Στοιχείων Ελέγχου με την τελική τιμή του ελέγχου καθώς και την χρονική διάρκειά του
		\item Ο χρήστης επιβεβαιώνει τα στοιχεία
		\item Το σύστημα μεταφέρει τον χρήστη στο μενού πληρωμών. \red{Μετά την επιτυχή πληρωμή, δημιουργείται η Απόδειξη της Συναλλαγής και προστίθενται σε αυτήν τα απαραίτητα στοιχεία}
		\red{\item Το σύστημα προσθέτει στην Οντότητα του Ελέγχου τις απαραίτητες πληροφορίες, και καταχωρεί τον Έλεγχο στην λίστα των εκκρεμών ελέγχων, του Ελεγκτή που επιλέχθηκε αλλά και στην λίστα των Ελέγχων, του συστήματος}
		\red{\item Το σύστημα καταχωρεί την συναλλαγή στο \en \textit{TransactionLog} \gr, ανακτά την Απόδειξη και αποστέλλει \en email \gr στον χρήστη, με τα στοιχεία του ραντεβού, του ελεγκτή και την απόδειξη της συναλλαγής. Τέλος, εμφανίζει την Οθόνη Επιτυχούς \red{Προγραμματισμού Ελέγχου}}		
	\end{enumerate}
	
	
	
	\paragraph{Εναλλακτική Ροή 1}
	
	\begin{enumerate}
		\item Ο χρήστης εισάγει μη-υπαρκτή τοποθεσία
		\item Το σύστημα εμφανίζει μήνυμα σφάλματος, επιστρέφει τον χρήστη στην οθόνη \textit{Προγραμματισμού Ελέγχου Οχήματος}, προτρέποντάς τον να εισάγει ξανά την τοποθεσία του
		\item Ο χρήστης εισάγει την σωστή τοποθεσία του
		\item Το σύστημα εντοπίζει τον χρήστη και η Περίπτωση Χρήσης προχωρά από το βήμα 4 της βασικής ροής
	\end{enumerate}
	
	\paragraph{Εναλλακτική Ροή 2}
	
	\begin{enumerate}
		\item Ο χρήστης απορρίπτει τον προτεινόμενο από το σύστημα ελεγκτή, προκειμένου να επιλέξει τον ελεγκτή της αρεσκείας του
		\item Το σύστημα εμφανίζει την Οθόνη Εισαγωγής Στοιχείων Ελεγκτή
		\item Ο χρήστης εισάγει τα στοιχεία του ελεγκτή
		\item Το σύστημα \red{εισάγει τα στοιχεία του Ελεγκτή στην Οντότητα του Ελέγχου, ελέγχοντας αν υπάρχει πράγματι εγγεγραμμένος ο εν λόγων Ελεγκτής.}
		\red{\item Το σύστημα επιστρέφει τον χρήστη στην οθόνη Στοιχείων Ελεγκτή και η Περίπτωση Χρήσης συνεχίζει από το βήμα 8 της Βασικής Ροής}		
	\end{enumerate}	

	\paragraph{\red{Ένθετη Εναλλακτική Ροή 2.1}}
	
	\begin{enumerate}
		\item \red{Τα στοιχεία που εισήγαγε ο χρήστης, δεν αντιστοιχούν σε εγγεγραμμένο στην πλατφόρμα, Ελεγκτή}
		\item \red{Το σύστημα εμφανίζει την οθόνη Απόρριψης Ελέγχου, ο έλεγχος ακυρώνεται και επιστρέφει τον χρήστη στο αρχικό μενού}
	\end{enumerate}
	
	
	
	
	\paragraph{\en Use Case 3: \gr Αναζήτηση Ανταλλακτικού}	
	
	\begin{enumerate}
		\item Ο χρήστης επιλέγει \en"\gr Αναζήτηση Ανταλλακτικού\en" \gr στο αρχικό μενού
		\item Το σύστημα εμφανίζει την οθόνη Εισαγωγή Χαρακτηριστικών Ανταλλακτικού
		\item Ο χρήστης περιορίζει την αναζήτηση του τοποθετώντας το είδος του οχήματος, την μάρκα, το μοντέλο, τον κατασκευαστή, το εύρος τιμών, και την κατάσταση του ανταλλακτικού (καινούργιο ή μεταχειρισμένο) 
		\item Το σύστημα ελέγχει πως συμπληρώθηκαν τα πεδία της αναζήτησης και στην συνέχεια ορίζει τις παραμέτρους της αναζήτησης\footnote[1]{Ο ορισμός των παραμέτρων της αναζήτησης γίνεται και στην βασική ροή και στην εναλλακτική ροή αλλά και στην ένθετη εναλλακτική ροή. Αυτό γίνεται καθώς ανάλογα με την περίπτωση, θα πρέπει να οριστούν διαφορετικές παράμετροι καθώς οι 3 αναζητήσεις διαφέρουν μεταξύ τους. Συνεπώς, δεν αποτελεί σχεδιαστικό πλεονασμό, μιας και είναι απαραίτητος ο διαχωρισμός των παραμέτρων ανάλογα με το είδος της αναζήτησης.}, με βάση τα κριτήρια που εισήγαγε ο χρήστης
		\item Το σύστημα ανακτά όλες τις αγγελίες ανταλλακτικών που πληρούν τα απαραίτητα κριτήρια, δημιουργεί τα Αποτελέσματα Αναζήτησης (εφήμερο αντικείμενο) και έπειτα εμφανίζει την οθόνη Αποτελέσματα Αναζήτησης, με την λίστα των αγγελιών
		\item Ο χρήστης επιλέγει την αγγελία της αρεσκείας του
		\item Το σύστημα μεταφέρει τον χρήστη στην οθόνη Αγγελία Ανταλλακτικού, εμφανίζοντας μια λεπτομερή περιγραφή του ανταλλακτικού και τα στοιχεία του πωλητή		
	\end{enumerate}
	
	\paragraph{Εναλλακτική Ροή}
	
	\begin{enumerate}
		\item Ο χρήστης δεν εισάγει χαρακτηριστικά για το ανταλλακτικό που επιθυμεί να αγοράσει
		\item Το σύστημα ανακτά το ιστορικό των αγορών οχημάτων του χρήστη και ελέγχει αν είναι κενό ή όχι. Σε περίπτωση που είναι μη-κενό, ορίζει τις παραμέτρους της αναζήτησης με βάση το ιστορικό αγορών του χρήστη και η Περίπτωση Χρήσης συνεχίζει από το βήμα 5 της βασικής ροής. Ειδάλλως, οδηγούμαστε στην \textit{Ένθετη Εναλλακτική Ροή}
		
	\end{enumerate}
	
	\paragraph{Ένθετη Εναλλακτική Ροή}
	\begin{enumerate}
			\item Ο χρήστης δεν έχει ιστορικό αγορών οχημάτων
			\item Το σύστημα ορίζει τις παραμέτρους της αναζήτησης, με τέτοιο τρόπο ώστε να εμφανιστούν όλες οι αγγελίες ανταλλακτικών. Η Περίπτωση Χρήσης συνεχίζει από το βήμα 5 της βασικής ροής
	\end{enumerate}
	
	
	\paragraph{\en Use Case 4: \gr Αναζήτηση Κοντινών Αντιπροσωπειών}	
	
	\begin{enumerate}
		\item Ο χρήστης επιλέγει  \en"\gr Εύρεση Κοντινών Αντιπροσωπειών\en" \gr στο αρχικό μενού
		\item Το σύστημα εμφανίζει την οθόνη Εισαγωγή Τοποθεσίας η οποία περιέχει ένα χάρτη και ένα πεδίο αναζήτησης, και προτρέπει τον χρήστη να εισάγει την περιοχή του και την ακτίνα αναζήτησης
		\item Ο χρήστης εισάγει την περιοχή του και την επιθυμητή ακτίνα αναζήτησης		
		\item Το σύστημα ελέγχει την εγκυρότητα της εισαχθείσας τοποθεσίας και εντοπίζει τον χρήστη. Στην συνέχεια, εμφανίζει την οθόνη Επιλογής Μάρκας Οχημάτων, προτρέποντας τον χρήστη να επιλέξει μάρκες οχημάτων που επιθυμεί να διαθέτουν οι κοντινές του αντιπροσωπείες
		\item Ο χρήστης επιλέγει τις μάρκες του ενδιαφέροντός του			
		\item Το σύστημα ελέγχει πως επιλέχθηκαν μάρκες οχημάτων και ανακτά την λίστα των καταστημάτων που πληρούν τα κριτήρια που έθεσε ο χρήστης
		\item Το σύστημα δημιουργεί τα αποτελέσματα της αναζήτησης και εμφανίζει την οθόνη Λίστα Αντιπροσωπειών, όπου περιέχονται τα αποτελέσματα της αναζήτησης του χρήστη
		\item Ο χρήστης επιλέγει την αντιπροσωπεία της αρεσκείας του
		\item Το σύστημα εμφανίζει την οθόνη Οχήματα Αντιπροσωπείας η οποία περιέχει μία λίστα με τα οχήματα που είναι διαθέσιμα από την αντιπροσωπεία     	
	\end{enumerate}
	
	\paragraph{Εναλλακτική Ροή 1}
	\begin{enumerate}
		\item Ο χρήστης δεν επιλέγει μάρκες οχημάτων στο βήμα 5
		\item Το σύστημα ανακτά καταστήματα αντιπροσωπειών που διαθέτουν οχήματα των εταιρειών των οχημάτων που βρίσκονται στην \en wishlist \gr του χρήστη, και η Περίπτωση Χρήσης συνεχίζεται από το βήμα 7 της βασικής ροής
	\end{enumerate}
	
	\paragraph{Εναλλακτική Ροή 2}
	\begin{enumerate}
			\item Ο χρήστης εισάγει μη-υπαρκτή τοποθεσία
			\item Το σύστημα εμφανίζει μήνυμα σφάλματος, επιστρέφει τον χρήστη στην οθόνη \textit{Εισαγωγή Τοποθεσίας}, προτρέποντάς τον να εισάγει ξανά την τοποθεσία του και η Περίπτωση Χρήσης συνεχίζει από το βήμα 4 της βασικής ροής		
	\end{enumerate}
	
	
	
	\paragraph{\en Use Case 5: \gr Σύγκριση Αυτοκινήτων}
	\begin{enumerate}
		\item Ο χρήστης επιλέγει \en"\gr Σύγκριση Αυτοκινήτων\en" \gr στο αρχικό μενού
		\item Το σύστημα εμφανίζει την οθόνη Σύγκρισης Αυτοκινήτων και προτρέπει τον χρήστη να εισάγει τους κωδικούς διαφορετικών αγγελιών, τα οχήματα των οποίων επιθυμεί να συγκρίνει
		\item Ο χρήστης εισάγει τους κωδικούς των αγγελιών
		\item Το σύστημα ελέγχει αν οι κωδικοί είναι έγκυροι και στην συνέχεια, αν είναι διαφορετικοί μεταξύ τους
		\item To σύστημα δημιουργεί την Σύγκριση (Οντότητα \en \textit{CarComparison}\gr), \red{προσθέτει σε αυτήν την λίστα των αγγελιών} και εμφανίζει την οθόνη Κριτήρια Αυτοκινήτων προς Σύγκριση, προτρέποντας τον χρήστη να εισάγει το επιθυμητό εύρος τιμών και τα σημαντικά κριτήρια που θα συντελέσουν στην επιλογή ενός οχήματος
		\item Ο χρήστης εισάγει το επιθυμητό εύρος τιμών και καθορίζει τα κυρίαρχα κριτήρια της σύγκρισης
		\item Το σύστημα εισάγει τις απαραίτητες πληροφορίες στην Οντότητα της Σύγκρισης. Έπειτα, εξετάζει τα οχήματα και επιλέγει το/τα προτεινόμενο/α όχημα/τα, με βάση τα κριτήρια που εισήγαγε ο χρήστης
		\item Το σύστημα δημιουργεί τα Αποτελέσματα της Σύγκρισης και προσθέτει την Σύγκριση στο \en \textit{CarListingsStatisticsLog} \gr. Έπειτα, εμφανίζει την οθόνη Αποτελέσματα Σύγκρισης, προβάλλοντας μία λίστα με τα αυτοκίνητα και τα χαρακτηριστικά που ο χρήστης επέλεξε να πάρουν μέρος στην σύγκριση, αλλά και το κόστος των τελών κυκλοφορίας και των ασφαλίστρων, καθώς και την πρόταση του συστήματος 
		\item Ο χρήστης επιλέγει το όχημα που επιθυμεί
		\item Το σύστημα ανακτά τα στοιχεία της αγγελίας και μεταφέρει τον χρήστη στην οθόνη Λεπτομέρειες Αγγελίας, επιτρέποντάς του να εξετάσει αναλυτικότερα το επιλεγμένο όχημα
	\end{enumerate}
	
	\paragraph{Εναλλακτική Ροή 1}
	
	\begin{enumerate}
		\item Ο χρήστης εισάγει κωδικούς που δεν είναι διαφορετικοί μεταξύ τους
		\item Το σύστημα εμφανίζει προειδοποιητικό μήνυμα και \red{ανακτά τα οχήματα που έχει αποθηκεύσει ο χρήστης στην \en Wishlist \gr του, αλλά και οχήματα που συμμετέχουν συχνά σε συγκρίσεις άλλων χρηστών (ανάκτηση από το \en \textit{CarListingsStatisticsLog}\gr). Η Περίπτωση Χρήσης συνεχίζει από το βήμα \red{8} της Βασικής Ροής με την δημιουργία των αποτελεσμάτων της σύγκρισης}
	\end{enumerate}
	
	\paragraph{Εναλλακτική Ροή 2}
	\begin{enumerate}
		\item Ο χρήστης εισάγει μη-έγκυρους κωδικούς αγγελιών οχημάτων
		\item Το σύστημα εμφανίζει προειδοποιητικό μήνυμα και επιστρέφει τον χρήστη στην οθόνη \textit{Σύγκριση Αυτοκινήτων}, προτρέποντάς τον να εισάγει ξανά κωδικούς αγγελιών
		\item Ο χρήστης εισάγει κωδικούς αγγελιών και η Περίπτωση Χρήσης συνεχίζει από το βήμα 4 της Βασικής Ροής		
	\end{enumerate}
	
	
	\paragraph{\en Use Case 6: \gr Προσθήκη Καταστήματος Αντιπροσωπείας}
	
	\begin{enumerate}
		\item Ο υπεύθυνος της αντιπροσωπείας επιλέγει \en"\gr Προσθήκη Καταστήματος\en" \gr στο αρχικό μενού
		\item Το σύστημα εμφανίζει την οθόνη Εισαγωγής Καταστήματος Αντιπροσωπείας, προτρέποντας τον χρήστη να εισάγει το όνομα της εταιρείας στην οποία υπάγεται το κατάστημα
		\item Ο υπεύθυνος εισάγει το όνομα της εταιρείας
		\item Το σύστημα επιβεβαιώνει πως στην Βάση Δεδομένων της πλατφόρμας, υπάρχει εγγεγραμμένη η αντίστοιχη εταιρεία
		\item Το σύστημα εμφανίζει τον χάρτη και ζητά από τον χρήστη να εισάγει την τοποθεσία του καταστήματος
		\item Ο υπεύθυνος της αντιπροσωπείας εισάγει τα λεπτομερή γεωγραφικά στοιχεία του καταστήματος
		\item Το σύστημα εντοπίζει το κατάστημα στον χάρτη. \red{Έπειτα, δημιουργεί το κατάστημα (Οντότητα \en \textit{DealershipStore}\gr), προσθέτει στην οντότητα τα απαραίτητα στοιχεία και ελέγχει αν υπάρχει ήδη καταχωρημένο το συγκεκριμένο κατάστημα}
		\red{\item Το σύστημα εμφανίζει την οθόνη Λεπτομέρειες Καταστήματος, ζητώντας από τον χρήστη να εισάγει τον τίτλο του καταστήματος και μια λίστα με τα αυτοκίνητα που διαθέτει προς πώληση} 
		\item Ο υπεύθυνος εισάγει τον τίτλο και τα οχήματα που διαθέτει το κατάστημα
		\item Το σύστημα μεταφέρει τον χρήστη στην οθόνη Επιβεβαίωσης, στην οποία εμφανίζονται τα στοιχεία του καταστήματος (όνομα, τοποθεσία) και η λίστα με τα αυτοκίνητα που διαθέτει
		\item Ο υπεύθυνος επιβεβαιώνει τα στοιχεία
		\red{\item Το σύστημα καταχωρεί το κατάστημα και εμφανίζει την οθόνη Δημιουργία Διαφήμισης, προτρέποντας τον χρήστη, να δημιουργήσει μια διαφήμιση για το συγκεκριμένο κατάστημα, με σκοπό την ενημέρωση των χρηστών της πλατφόρμας που βρίσκονται στην περιοχή του καταστήματος}
		\item Ο υπεύθυνος δημιουργεί την σχετική διαφήμιση
		\item Το σύστημα δημιουργεί την διαφήμιση, \red{προσθέτει σε αυτήν τα απαραίτητα στοιχεία} και εμφανίζει μήνυμα επιτυχούς προσθήκης καταστήματος
	\end{enumerate}
	
	\paragraph{Εναλλακτική Ροή 1}
	
	\begin{enumerate}
		\item Ο υπεύθυνος της αντιπροσωπείας εισάγει όνομα εταιρείας, η οποία δεν ανήκει στην πλατφόρμα
		\item Το σύστημα εμφανίζει προειδοποιητικό μήνυμα και μεταφέρει τον χρήστη στην οθόνη Εγγραφή Εταιρείας, προτρέποντας τον χρήστη να εγγράψει στην πλατφόρμα την εταιρεία με το όνομα που εισήγαγε		
		\item Ο υπεύθυνος εγγράφει την εταιρεία εισάγοντας τα απαραίτητα στοιχεία
		\item Το σύστημα καταχωρεί την εταιρεία στις ήδη εγγεγραμμένες και η Περίπτωση Χρήσης συνεχίζει από το βήμα 5 της βασικής ροής
	\end{enumerate}
	
	\paragraph{Εναλλακτική Ροή 2}
	
\begin{enumerate}
			\item Ο υπεύθυνος της αντιπροσωπείας επιχειρεί να προσθέσει κατάστημα, το οποίο είναι ήδη καταχωρημένο στην πλατφόρμα
			\item Το σύστημα εμφανίζει μήνυμα σφάλματος και μεταφέρει τον χρήστη πίσω στο αρχικό μενού
	\end{enumerate}
	
	
	\paragraph{\en Use Case 7: \gr Προγραμματισμός \en Test Drive \gr}
	
	\begin{enumerate}
		\item Ο χρήστης επιλέγει \en"Test Drive" \gr στο αρχικό μενού
		\item Το σύστημα εμφανίζει την οθόνη Καταχώρησης Κωδικού Αγγελίας Οχήματος
		\item Ο χρήστης εισάγει τον κωδικό της αγγελίας, για το όχημα της οποίας ενδιαφέρεται για \en Test Drive \gr
		\item Το σύστημα ελέγχει πως ο κωδικός αντιστοιχεί σε καταχωρημένη αγγελία, ανακτά τα στοιχεία της,  
		\red{δημιουργεί την Οντότητα του \en Test Drive \gr και θέτει σε αυτήν τις απαραίτητες πληροφορίες (εκτός της ημερομηνίας)}
		\red{\item Το σύστημα μεταφέρει τον χρήστη στην οθόνη Προγραμματισμού \en Test Drive \gr }
		\item Ο χρήστης εισάγει την επιθυμητή ημερομηνία και ώρα		
		\item Το σύστημα \red{θέτει στην Οντότητα του \en Test Drive \gr, την επιλεγμένη ημερομηνία και έπειτα ελέγχει αν η ημερομηνία και ώρα είναι διαθέσιμες}. Στην συνέχεια εμφανίζει την οθόνη Στοιχεία Ραντεβού, με τις λεπτομέρειες του ραντεβού
		\item Ο χρήστης επιβεβαιώνει τα στοιχεία
		\item \red{Το σύστημα καταχωρεί το \en Test Drive \gr στην πλατφόρμα} και αποστέλλει στο \en email \gr του χρήστη και του πωλητή του οχήματος, τα λεπτομερή στοιχεία του ραντεβού. Τέλος, εμφανίζει την οθόνη Επιτυχούς προγραμματισμού \en Test Drive \gr 
	\end{enumerate}
	
	\paragraph{Εναλλακτική Ροή 1}
	
	\begin{enumerate}
		\item Ο χρήστης επιλέγει μη-διαθέσιμη ημερομηνία και ώρα
		\item Το σύστημα εμφανίζει μήνυμα σφάλματος σχετικά με την μη-διαθεσιμότητα της επιλεγμένης ημερομηνίας και μεταφέρει τον χρήστη στην οθόνη \textit{Προγραμματισμού \en Test Drive \gr}, προτρέποντάς τον επιλέξει ξανά
		\item Ο χρήστης επιλέγει νέα ημερομηνία και ώρα και η Περίπτωση Χρήσης συνεχίζει από το βήμα 6 της βασικής ροής
	\end{enumerate}
	
	
	\paragraph{Εναλλακτική Ροή 2}
	
	\begin{enumerate}
		\item Ο χρήστης εισάγει κωδικό μη-καταχωρημένης αγγελίας
		\item Το σύστημα εμφανίζει σχετικό μήνυμα σφάλματος και μεταφέρει τον χρήστη στην οθόνη \textit{Καταχώρησης Κωδικού Αγγελίας Οχήματος}, προτρέποντάς τον να επανεισάγει τον κωδικό
		\item Ο χρήστης εισάγει τον κωδικό της αγγελίας και η Περίπτωση Χρήσης συνεχίζει από το βήμα 4 της βασικής ροής
	\end{enumerate}
	
	
	\paragraph{\en Use Case 8: \gr  Ανταλλαγή Οχήματος \gr}
	
	\begin{enumerate}
		\item Ο χρήστης επιλέγει \en"\gr Ανταλλαγή Οχήματος\en" \gr στο αρχικό μενού		
		\item Το σύστημα εμφανίζει την οθόνη Ανταλλαγής Οχήματος, προτρέποντας τον χρήστη να εισάγει τα χαρακτηριστικά του οχήματος \red{ και δημιουργεί την Οντότητα του Αυτοκινήτου (\en \textit{Car}\gr)}		
		\item Ο χρήστης εισάγει τα χαρακτηριστικά του οχήματος που επιθυμεί να αποσύρει
		\red{\item Το σύστημα προσθέτει στην οντότητα του αυτοκινήτου, τα στοιχεία του οχήματος και ελέγχει πως όντως κυκλοφορεί αντίστοιχο μοντέλο αυτοκινήτου στην αγορά }
		\red{\item Το σύστημα υπολογίζει την αξία του οχήματος, με βάση τα χαρακτηριστικά και την κατάστασή του, δημιουργεί την Οντότητα Ανταλλαγή Οχήματος (\en \textit{CarExchange} \gr) και προσθέτει σε αυτή τα στοιχεία του οχήματος}
		\item Το σύστημα βρίσκει τις αντιπροσωπείες που δέχονται το συγκεκριμένο όχημα και μεταφέρει τον χρήστη στην οθόνη \red{Επιλογή Καταστήματος Αντιπροσωπείας}, όπου εμφανίζεται μια λίστα με τα καταστήματα με την προσφορά της κάθε αντιπροσωπείας, ως αντάλλαγμα για το όχημα
		\item Ο χρήστης επιλέγει ένα κατάστημα αντιπροσωπείας
		\red{\item Το σύστημα ανακτά τα στοιχεία του επιλεγμένου καταστήματος και προσθέτει στην Οντότητα της Ανταλλαγής, το εν λόγω κατάστημα}
		\red{\item Το σύστημα εμφανίζει την οθόνη Ανάρτησης Νομικών Εγγράφων, προτρέποντας στον χρήστη να αναρτήσει τα απαραίτητα νομικά έγγραφα του οχήματος}
		\item Ο χρήστης αναρτά τα απαραίτητα έγγραφα		
		\item Το σύστημα προσθέτει τα έγγραφα στην Οντότητα της Ανταλλαγής και εμφανίζει την οθόνη Ολοκλήρωσης Ανταλλαγής, στην οποία αναγράφονται οι λεπτομέρειες της ανταλλαγής, και ζητά από τον χρήστη να επιβεβαιώσει την αποδοχή της ανταλλαγής
		\item Ο χρήστης αποδέχεται την ανταλλαγή
		\red{\item Το σύστημα δημιουργεί την αντίστοιχη Συναλλαγή, προσθέτει σε αυτή τις απαραίτητες πληροφορίες και την προσθέτει στην Οντότητα της Ανταλλαγής. Έπειτα, καταχωρεί την Συναλλαγή στο \en\textit{TransactionLog} \gr}
		\red{\item Το σύστημα ανακτά τα νομικά έγγραφα της Ανταλλαγής, αποστέλλει \en email \gr στον χρήστη με τα νομικά έγγραφα της ανταλλαγής και εμφανίζει μήνυμα επιτυχούς ανταλλαγής}
	\end{enumerate}
	
	\paragraph{{Εναλλακτική Ροή}}
	
	\begin{enumerate}
		\item Ο χρήστης εισάγει στοιχεία μη-υπαρκτού οχήματος
		\item Το σύστημα εμφανίζει προειδοποιητικό μήνυμα μη-υπαρκτού οχήματος και μεταφέρει τον χρήστη στην οθόνη \textit{Ανταλλαγή Οχήματος}, προτρέποντάς τον να εισάγει ξανά τα χαρακτηριστικά του αυτοκινήτου
		\item Ο χρήστης εισάγει τα χαρακτηριστικά και η Περίπτωση Χρήσης συνεχίζει από το βήμα 4 της βασικής ροής
	\end{enumerate}
	
	
	\paragraph{\en Use Case 9: \gr Αγορά Οχήματος\gr}
	
	\begin{enumerate}
		\item Ο χρήστης επιλέγει \en"\gr Αγορά Οχήματος\en" \gr στο αρχικό μενού
		\item Το σύστημα αποστέλλει έναν κωδικό ασφαλείας στο \en email \gr του χρήστη και εμφανίζει την οθόνη Εισαγωγής Κωδικού Ασφαλείας
		\item Ο χρήστης εισάγει τον κωδικό ασφαλείας		
		\item Το σύστημα ελέγχει την εγκυρότητα του Κωδικού Ασφαλείας και εμφανίζει την οθόνη Εισαγωγής Κωδικού Αγγελίας Οχήματος
		\item Ο χρήστης εισάγει τον κωδικό της αγγελίας	του οχήματος που επιθυμεί να αγοράσει
		\item Το σύστημα ελέγχει πως ο δοσμένος κωδικός αντιστοιχεί σε καταχωρημένη αγγελία, \red{εμφανίζει την οθόνη Διαχείρισης Οικονομικών και ανακτά την Αγγελία του Οχήματος. Το σύστημα προτρέπει τον χρήστη να επιλέξει αν επιθυμεί να πληρώσει με άτοκες δόσεις}
		\item Ο χρήστης επιλέγει \red{τον τρόπο πληρωμής}
		\item Το σύστημα \red{ελέγχει αν επιλέχθηκε πληρωμή μέσω δόσεων}, εμφανίζει την οθόνη του Οικονομικού Συμβούλου και ζητά από τον χρήστη να εισάγει τον μηνιαίο μισθό του, με σκοπό τον υπολογισμό ενός προσαρμοσμένου στον χρήστη, ποσού άτοκης μηνιαίας δόσης
		\red{\item Το σύστημα δημιουργεί την Οντότητα της Μηνιαίας Δόσης (\en \textit{MonthlyInstallment}\gr), ανακτά από την Αγγελία, την τιμή του οχήματος και την προσθέτει στην Μηνιαία Δόση}		
		\item Ο χρήστης εισάγει τον μηνιαίο μισθό του
		\red{\item Το σύστημα υπολογίζει και εμφανίζει το ποσό της εξατομικευμένης μηνιαίας δόσης}
		\item Ο χρήστης αποδέχεται το ποσό της μηνιαίας δόσης
		\red{\item Το σύστημα επιστρέφει τον χρήστη στην οθόνη Διαχείρισης Οικονομικών}
		\red{\item Το σύστημα ανακτά τα στοιχεία του οχήματος και υπολογίζει τα τέλη κυκλοφορίας του}
		\red{\item Το σύστημα μεταφέρει τον χρήστη στην οθόνη Ολοκλήρωσης Αγοράς, όπου εμφανίζεται η τιμή του οχήματος, ο κωδικός της αγγελίας, το όνομα του οχήματος, τα τέλη κυκλοφορίας του καθώς και το ποσό της μηνιαίας δόσης σε περίπτωση που η πληρωμή θα γίνει μέσω άτοκων δόσεων}
		\red{\item Ο χρήστης επιβεβαιώνει την ορθότητα των στοιχείων της αγοράς}		
		\red{\item Το σύστημα μεταφέρει τον χρήστη στην σελίδα του συστήματος πληρωμών. Μετά την επιτυχή πληρωμή, δημιουργεί την Οντότητα της Απόδειξης της Συναλλαγής (\en \textit{Invoice}\gr), εισάγει σε αυτή τις απαραίτητες πληροφορίες και προσθέτει την Συναλλαγή στην Οντότητα της Μηνιαίας Δόσης}
		\red{\item Το σύστημα καταχωρεί την Συναλλαγή στο \en \textit{TransactionLog} \gr, προσθέτει το όχημα στην λίστα των αγορών του Χρήστη, ανακτά τα στοιχεία της Απόδειξης και αποστέλλει στο \en email \gr του χρήστη την Απόδειξη της Συναλλαγής. Τέλος, εμφανίζει την Οθόνη Επιτυχούς Αγοράς Οχήματος}
	\end{enumerate}
	
	\paragraph{Εναλλακτική Ροή 1}
	\begin{enumerate}
		\item Ο χρήστης, \red{στο βήμα 7 δεν επιλέγει πληρωμή μέσω άτοκων δόσεων} και η Περίπτωση Χρήσης συνεχίζει από το βήμα \red{14} της βασικής ροής, με την ανάκτηση των στοιχείων του οχήματος
	\end{enumerate}
	
	\paragraph{Εναλλακτική Ροή 2}
	\begin{enumerate}
		\item Ο χρήστης εισάγει κωδικό μη-υπαρκτής αγγελίας
		\item Το σύστημα εμφανίζει μήνυμα μη-υπαρκτής αγγελίας και επιστρέφει τον χρήστη στην οθόνη \textit{Εισαγωγή Κωδικού Αγγελίας Οχήματος}, προκειμένου να εισάγει ξανά τον κωδικό της αγγελίας 
	\end{enumerate}
	
	\paragraph{Εναλλακτική Ροή 3}
	\begin{enumerate}
			\item Ο χρήστης εισάγει κωδικό ασφαλείας διαφορετικό από αυτόν που στάλθηκε στο \en email \gr του
			\item Το σύστημα εμφανίζει μήνυμα σφάλματος και επιστρέφει τον χρήστη στο αρχικό μενού, ακυρώνοντας την αγορά
	\end{enumerate}
	
	
	\paragraph{\en Use Case 10: \gr Αναζήτηση Οχήματος}  
	\begin{enumerate}
		\item Ο χρήστης επιλέγει \en"\gr Αναζήτηση Οχήματος\en" \gr στο αρχικό μενού
		\item Το σύστημα εμφανίζει την οθόνη \en"\gr Χαρακτηριστικά Οχήματος\en" \gr, προτρέποντας τον χρήστη να εισάγει τα χαρακτηριστικά του οχήματος 
		\item Ο χρήστης συμπληρώνει όσα πεδία επιθυμεί
		\item Το σύστημα δημιουργεί την αναζήτηση (Οντότητα \en \textit{CarSearch}\gr). Έπειτα εμφανίζει την οθόνη Φίλτρα Αναζήτησης, δίνοντας στον χρήστη την επιλογή να επιλέξει ανάμεσα σε αγγελίες ιδιωτών ή/και αντιπροσωπειών, να καθορίσει τον κριτήριο ταξινόμησης των αγγελιών, να εισάγει την τοποθεσία του και την ακτίνα αναζήτησης, καθώς και το εύρος τιμών εντός του οποίου πρέπει να κυμαίνονται τα αποτελέσματα
		\item Ο χρήστης συμπληρώνει τα πεδία και εισάγει την τοποθεσία του και την επιθυμητή ακτίνα αναζήτησης
		\item Το σύστημα ελέγχει αν συμπληρώθηκαν τα πεδία, \red{προσθέτει στην Οντότητα της Αναζήτησης Οχήματος, τα απαραίτητα στοιχεία και εντοπίζει τον χρήστη }
		\red{\item Το σύστημα δημιουργεί τα αποτελέσματα της αναζήτησης, καταχωρεί την αναζήτηση στο \en \textit{CarListingsStatisticsLog} \gr, και εμφανίζει την οθόνη \en"\gr Αποτελέσματα Αναζήτησης\en" \gr, με την λίστα των αγγελιών }
		\item Ο χρήστης επιλέγει μια αγγελία με σκοπό να δει λεπτομέρειες για το όχημα
		\item Το σύστημα \red{ανακτά την Αγγελία του Οχήματος και } μεταφέρει τον χρήστη στην οθόνη \en"\gr Λεπτομέρειες Αγγελίας\en" \gr, με σκοπό την προβολή περαιτέρω πληροφοριών	
	\end{enumerate}
	
	
	\paragraph{Εναλλακτική Ροή}
		\begin{enumerate}
			\item Ο χρήστης δεν συμπληρώνει τα πεδία της αναζήτησης
			\item Το σύστημα ψάχνει στο \en \textit{CarListingsStatisticsLog} \gr, τα οχήματα που συμμετέχουν συχνά στις αναζητήσεις των άλλων χρηστών και η Περίπτωση Χρήσης συνεχίζει από τη βήμα 7 της βασικής ροής
	\end{enumerate}
	
	
	
	\paragraph{\en Use Case 11: \gr Ανάρτηση Αγγελίας Πώλησης Ανταλλακτικού \gr}
	
	\begin{enumerate}
		\item Ο χρήστης επιλέγει \en"\gr Ανάρτηση Αγγελίας Ανταλλακτικού\en" \gr στο αρχικό μενού
		\item Το σύστημα εμφανίζει την οθόνη Ανάρτησης Αγγελίας Ανταλλακτικού
		\item Ο χρήστης εισάγει την τοποθεσία του και τον τίτλο της αγγελίας		
		\item Το σύστημα εντοπίζει τον χρήστη, \red{δημιουργεί την αγγελία του Ανταλλακτικού (Οντότητα \en\textit{SparePartListing}\gr) και προσθέτει σε αυτήν στοιχεία όπως ο τίτλος, η τοποθεσία και ο δημιουργός της αγγελίας}
		\red{\item Το σύστημα δημιουργεί την οντότητα του Ανταλλακτικού (\en \textit{SparePart}\gr) και εμφανίζει την οθόνη Εισαγωγής Στοιχείων Ανταλλακτικού		}
		\item Ο χρήστης εισάγει στοιχεία του ανταλλακτικού όπως η κατάστασή του (καινούριο ή μεταχειρισμένο), τον τύπο του, τον κωδικό του, την εταιρεία, το μοντέλο και την τιμή του		
		\item \red{Το σύστημα προσθέτει στην Οντότητα του Ανταλλακτικού, τις απαραίτητες πληροφορίες και} ελέγχει πως όντως υπάρχει ανταλλακτικό με τον δοσμένο κωδικό		
		\red{\item Το σύστημα τοποθετεί το ανταλλακτικό στην κατάλληλη κατηγορία με βάση τον κωδικό του, και προσθέτει στην Αγγελία, το Ανταλλακτικό, την κατάστασή του αλλά και την τιμή του}
		\red{\item Το σύστημα εμφανίζει την οθόνη Εισαγωγής Φωτογραφιών και Περιγραφής Ανταλλακτικού}		
		\item Ο χρήστης προσθέτει το κείμενο της περιγραφής και αναρτά τις φωτογραφίες του ανταλλακτικού		
		\item Το σύστημα ελέγχει πως προστέθηκε περιγραφή και αναρτήθηκαν φωτογραφίες. \red{Έπειτα, προσθέτει τις φωτογραφίες και την περιγραφή, στην Οντότητα της Αγγελίας του Ανταλλακτικού και εμφανίζει την οθόνη Προεπισκόπησης Αγγελίας}		
		\item Ο χρήστης εγκρίνει την αγγελία		
		\item Το σύστημα καταχωρεί την αγγελία και εμφανίζει οθόνη επιτυχούς καταχώρησης
	\end{enumerate}
	
	
	\paragraph{Εναλλακτική Ροή 1}
	
	\begin{enumerate}
		\item Ο χρήστης εισάγει κωδικό μη-υπαρκτού ανταλλακτικού
		\item Το σύστημα εμφανίζει προειδοποιητικό μήνυμα και επιστρέφει τον χρήστη στην Οθόνη \textit{Εισαγωγή στοιχείων Ανταλλακτικού}
		\item Ο χρήστης επανεισάγει τον κωδικό και η Περίπτωση Χρήσης προχωρά από το βήμα \red{6} της βασικής ροής
	\end{enumerate}
	
	\paragraph{Εναλλακτική Ροή 2}
	
	\begin{enumerate}
		\item Ο χρήστης δεν εισάγει περιγραφή ή δεν αναρτά φωτογραφίες του ανταλλακτικού
		\item Το σύστημα εμφανίζει προειδοποιητικό μήνυμα, και επιστρέφει τον χρήστη στην οθόνη \textit{Εισαγωγή Φωτογραφιών και Περιγραφής Ανταλλακτικού}, προτρέποντάς τον να συμπληρώσει τα αντίστοιχα πεδία
		\item Ο χρήστης εισάγει τις απαραίτητες ελλείπουσες πληροφορίες και η Περίπτωση Χρήσης συνεχίζει από το βήμα \red{11} της βασικής ροής
	\end{enumerate}
	
	
	\paragraph{\red{\en Use Case 12: \gr Έλεγχος Αναφοράς Αγγελίας}}
	\begin{enumerate}
		\item Ο υπάλληλος της Ασφαλιστικής εταιρείας επιλέγει \en"\gr Έλεγχος Αναφοράς\en" \gr στο αρχικό μενού
		\item Το σύστημα εμφανίζει την οθόνη Καταχωρημένων Αναφορών, η οποία περιέχει την λίστα με τις αναφορές, τον κωδικό τους και την κατάστασή τους (\en"\gr σε εκκρεμότητα \en" \gr ή \en"\gr ελεγμένη\en"\gr)
		\item Ο υπάλληλος επιλέγει ή εισάγει τον κωδικό της αναφοράς που επιθυμεί να ελέγξει 
		\item Το σύστημα ελέγχει την εγκυρότητα του κωδικού αναφοράς και ανακτά την αναφορά αγγελίας
		\item Το σύστημα μεταφέρει τον χρήστη στην οθόνη Λεπτομέρειες Αναφοράς, εμφανίζοντας τον δημιουργό την αναφοράς, την ημερομηνία αλλά και την αιτία δημιουργίας της, καθώς και την αγγελία που αποτελεί αντικείμενο της αναφοράς, ενώ δίνει στον χρήστη την επιλογή να διαγράψει ή όχι την αγγελία
		\item Ο υπάλληλος της εταιρείας, εξετάζει την αναφορά και επιλέγει ενέργεια (διαγραφή ή όχι)
		\item Το σύστημα ελέγχει αν επιλέχθηκε διαγραφή της αγγελίας και μεταφέρει τον χρήστη στην οθόνη Διαγραφής Αγγελίας \footnote[2]{Θεωρούμε πως στην πλειοψηφία των περιπτώσεων, ο υπάλληλος θα προχωρήσει σε διαγραφή της αγγελίας. Δηλαδή, πως συνήθως, οι αναφορές είναι βάσιμες}
		\item Ο υπάλληλος συμπληρώνει την φόρμα, με την αιτία διαγραφής της αγγελίας, και προχωρά στην υποβολή της
		\red{\item Το σύστημα δημιουργεί την Φόρμα Διαγραφής της αγγελίας (Οντότητα \en \textit{ListingDeletionForm}\gr) και προσθέτει σε αυτήν τις απαραίτητες πληροφορίες}		
		\item Το σύστημα διαγράφει την Αγγελία, \red{ανακτά τα στοιχεία της Φόρμας Διαγραφής} και αποστέλλει \en email \gr στον δημιουργό της αγγελίας, με ένα αντίγραφο της φόρμας Διαγραφής Αγγελίας που δημιούργησε ο υπάλληλος της εταιρείας. Τέλος, σημειώνει την αναφορά ως \textit{Ελεγμένη} και επιστρέφει τον υπάλληλο στο αρχικό μενού
	\end{enumerate}
	
	\paragraph{Εναλλακτική Ροή 1}
	\begin{enumerate}
		\item Ο υπάλληλος της ασφαλιστικής εταιρείας εισάγει κωδικό αναφοράς που δεν αντιστοιχεί σε κάποια καταχωρημένη αναφορά
		\item Το σύστημα εμφανίζει προειδοποιητικό μήνυμα και τον μεταφέρει στην οθόνη \textit{Καταχωρημένων Αναφορών}, προτρέποντάς τον να εισάγει ξανά τον κωδικό ή να επιλέξει μια αναφορά και η Περίπτωση Χρήσης συνεχίζει από το βήμα 3 της βασικής ροής
	\end{enumerate}	
	
	\paragraph{Εναλλακτική Ροή 2}
	\begin{enumerate}
		\item Ο υπάλληλος της ασφαλιστικής εταιρείας κρίνει πως η αγγελία δεν παραβιάζει κάποιον όρο της πλατφόρμας και δεν χρειάζεται να διαγραφεί
		\item Το σύστημα σημειώνει την αναφορά ως \textit{Ελεγμένη}	και επιστρέφει τον χρήστη στο αρχικό μενού
	\end{enumerate}
	
	\paragraph{\en Use Case 13: \gr Αγορά Ασφαλιστικού Πακέτου}
	\begin{enumerate}
		\item Ο χρήστης επιλέγει \en"\gr Αγορά Ασφαλιστικού Πακέτου\en" \gr στο αρχικό μενού
		\item Το σύστημα εμφανίζει την οθόνη Εισαγωγής Κωδικού Συναλλαγής
		\item Ο χρήστης εισάγει τον κωδικό της συναλλαγής, για το όχημα της οποίας επιθυμεί να αγοράσει ασφαλιστική κάλυψη
		\item Το σύστημα ελέγχει την εγκυρότητα του κωδικού συναλλαγής, και μεταφέρει τον χρήστη στην οθόνη Επιλογής Ασφαλιστικού Πακέτου, στην οποία εμφανίζονται τα στοιχεία της αγοράς του οχήματος. Το σύστημα προτρέπει τον χρήστη να επιλέξει το ασφαλιστικό πακέτο που επιθυμεί και να εισάγει το ποσό των πόντων που επιθυμεί να εξαργυρώσει, με σκοπό την εξασφάλιση έκπτωσης στα ασφάλιστρα \footnote[3]{Σε περίπτωση που ο χρήστης δεν επιθυμεί να εξαργυρώσει πόντους, μπορεί να εισάγει την τιμή \textbf{μηδέν} ως ποσό πόντων προς εξαργύρωση}
		\item Ο χρήστης επιλέγει ασφαλιστικό πακέτο και εισάγει το επιθυμητό ποσό πόντων προς εξαργύρωση
		\red{\item Το σύστημα ελέγχει αν ο χρήστης διαθέτει τον εν λόγω αριθμό πόντων και έπειτα			προχωρά στην εξαργύρωση των πόντων }	
		\red{\item Το σύστημα δημιουργεί την Οντότητα του Ασφαλιστικού Πακέτου (\en \textit{InsurancePlan}\gr), υπολογίζει την τιμή των ασφαλίστρων και προσθέτει στην Οντότητα τα απαραίτητα στοιχεία. Έπειτα, εμφανίζει την οθόνη Τιμή Ασφαλίστρων, όπου περιέχεται η τελική τιμή }
		\item Ο χρήστης αποδέχεται τα ασφάλιστρα
		\item Το σύστημα μεταφέρει τον χρήστη στο μενού πληρωμών. Μετά την επιτυχή πληρωμή, \red{το σύστημα δημιουργεί την Οντότητα της Απόδειξης της Συναλλαγής (\en \textit{Invoice}\gr) και προσθέτει σε αυτήν τις απαραίτητες πληροφορίες}
		\red{\item Το σύστημα καταχωρεί την συναλλαγή στο \en \textit{TransactionLog} \gr, ανακτά την Απόδειξη της Συναλλαγής και στέλνει \en email \gr στον χρήστη με την απόδειξη της πληρωμής και τα στοιχεία του συμβολαίου. Τέλος, εμφανίζει οθόνη επιτυχούς αγοράς}
	\end{enumerate}
	
	\paragraph{Εναλλακτική Ροή 1 }
	
	\begin{enumerate}
		\item Ο χρήστης εισάγει κωδικό μη-καταγεγραμμένης συναλλαγής
		\item Το σύστημα εμφανίζει μήνυμα σφάλματος και επιστρέφει τον χρήστη στην  οθόνη \textit{Εισαγωγή Κωδικού Συναλλαγής}, προτρέποντάς τον να εισάγει έγκυρο κωδικό συναλλαγής
		\item Ο χρήστης επανεισάγει τον κωδικό και η Περίπτωση Χρήσης συνεχίζει από το βήμα 4 της βασικής ροής
	\end{enumerate}
	
	\paragraph{Εναλλακτική Ροή 2}
		\begin{enumerate}
			\item Ο χρήστης εισάγει παραπάνω πόντους από όσους έχει στην κατοχή του
			\item Το σύστημα εμφανίζει μήνυμα σφάλματος ενημερώνοντας τον χρήστη πως δεν διαθέτει το συγκεκριμένο πόσο πόντων.
			\item Το σύστημα επιστρέφει τον χρήστη στην οθόνη \textit{Επιλογής Ασφαλιστικού Πακέτου} και η Περίπτωση Χρήσης, συνεχίζει από το βήμα 5 της Βασικής Ροής
	\end{enumerate}
	
	\paragraph{\en Use Case 14: \gr Μεταφορά Οχήματος}  
	\begin{enumerate}
		\item Ο χρήστης επιλέγει \en"\gr Μεταφορά Οχήματος\en" \gr στο αρχικό μενού
		\item Το σύστημα εμφανίζει την οθόνη Εισαγωγής Κωδικού Συναλλαγής και προτρέπει τον χρήστη να εισάγει τον κωδικό συναλλαγής
		\item Ο χρήστης εισάγει τον κωδικό που του είχε σταλεί μετά την ολοκλήρωση της αγοράς του οχήματος
		\item Το σύστημα αφού ελέγξει την εγκυρότητα του κωδικού συναλλαγής, εμφανίζει την οθόνη Εισαγωγής Τοποθεσίας Παράδοσης Οχήματος, ώστε να εισάγει ο χρήστης το επιθυμητό σημείο παράδοσης
		\item Ο χρήστης εισάγει την τοποθεσία που επιθυμεί
		\item Το σύστημα εντοπίζει τον χρήστη. Έπειτα, εμφανίζει την οθόνη Επιλογής Υπηρεσίας Μεταφοράς, προτρέποντας τον χρήστη να επιλέξει την υπηρεσία μεταφοράς που επιθυμεί (\en express \gr παράδοση ή κανονική)		
		\item Ο χρήστης επιλέγει την υπηρεσία της αρεσκείας του
		\red{\item Το σύστημα δημιουργεί την Οντότητα της Μεταφοράς Οχήματος (\en\textit{CarTransportation}\gr), προσθέτει σε αυτήν τα απαραίτητα στοιχεία και αναζητά Μεταφορέα με βάση την τοποθεσία στην οποία βρίσκεται το προς μεταφορά όχημα}				
		\item Το σύστημα μεταφέρει τον χρήστη στην οθόνη \en"\gr Στοιχεία Μεταφοράς\en"\gr, εμφανίζοντας τα έξοδα της μεταφοράς, τα στοιχεία του Μεταφορέα, το σημείο παράδοσης, τα στοιχεία του οχήματος καθώς και τον εκτιμώμενο χρόνο παράδοσης 
		\item Ο χρήστης επιβεβαιώνει την ορθότητα των στοιχείων 		
		\red{\item Το σύστημα μεταφέρει τον χρήστη στην οθόνη του συστήματος πληρωμών. Μετά την επιτυχή συναλλαγή, δημιουργεί την Οντότητα της Απόδειξης της Συναλλαγής (\en \textit{Invoice}\gr) και προσθέτει σε αυτήν τις απαραίτητες πληροφορίες  }
		\item \red{Το σύστημα καταχωρεί την Μεταφορά του Οχήματος και την προσθέτει στην λίστα των εκκρεμών μεταφορών του Μεταφορέα.} Έπειτα, καταγράφει την Συναλλαγή στο \en \textit{TransactionLog} \gr, \red{ανακτά την Απόδειξη της Συναλλαγής} και αποστέλλει \en email \gr στον Μεταφορέα αλλά και στον χρήστη με την απόδειξη της συναλλαγής. Τέλος, εμφανίζει μήνυμα επιτυχούς προγραμματισμού μεταφοράς
	\end{enumerate}
	
	\paragraph{Εναλλακτική Ροή}
	\begin{enumerate}
		\item Ο χρήστης εισάγει κωδικό μη-καταγεγραμμένης συναλλαγής
		\item Το σύστημα εμφανίζει μήνυμα μη-έγκυρου κωδικού συναλλαγής και επιστρέφει τον χρήστη στην οθόνη \textit{Εισαγωγή Κωδικού Συναλλαγής} 
		\item Ο χρήστης επανεισάγει τον κωδικό και η Περίπτωση Χρήσης συνεχίζει από το βήμα 3 της βασικής ροής
	\end{enumerate}
	
	
	\newpage
	
	\centering{ \textbf{\en  UML Use case Diagram \gr}}
	
	\vspace{25pt}
	
	\flushleft
	
	Στο \en Use Case Diagram \gr του έργου, βλέπουμε τους \en Actors \gr και τα \en Use Cases \gr που προκύπτουν από το Τεχνικό Κείμενο του \en Project Description \gr.
	
	\begin{figure}[htbp!]
		\includegraphics[scale=0.23]{img/UML_Use_case_diagram.png}
		\caption{\en UML Use Case Diagram \gr}
	\end{figure}
	
	
	
	
\end{document}



